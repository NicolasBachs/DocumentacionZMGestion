\chapter{Introducción}
\label{ch:capitulo1} 
\markboth{CAPÍTULO \ref*{ch:capitulo1}. Introducción}{CAPÍTULO \ref*{ch:capitulo1}. Introducción}
\section{Introducción}
\paragraph\indent
El presente documento corresponde al trabajo de graduación de la carrera de Ingeniería en Computación de la Facultad de Ciencias Exactas y Tecnología - Universidad Nacional de Tucumán, de los alumnos Nicolás Bachs y Loïk Choua.

Esta documentación surgió a partir de sucesivas versiones del mismo que fueron revisadas por los usuarios finales del sistema, los tutores y jurados del proyecto.

\section{Objetivos}
\subsection{Generales}
\paragraph\indent
Con respecto al trabajo de graduación, el objetivo principal es desarrollar el sistema de gestión para una mueblería, poniendo en práctica todos los conocimientos y habilidades adquiridas durante el transcurso de la carrera.

\subsection{Específicos}
\begin{itemize}
	\item Desarrollar el sistema de manera tal que, para los empleados actuales de la mueblería, sea intuitivo, fácil de utilizar y se pueda implementar rápidamente.
	\item Desarrollar un sistema que cumpla con los requisitos de software ANSI/IEEE 830 que son descriptos posteriormente en este documento.
	\item Desarrollar un sistema que permita automatizar los procesos de la mueblería.
	\item Desarrollar un sistema que permita brindar información resumida y procesada acerca de las operaciones que se llevan a cabo en la mueblería.
	\item Desarrollar un sistema que permita almacenar y procesar datos que contribuyan a la toma de decisiones en la mueblería.
\end{itemize}
