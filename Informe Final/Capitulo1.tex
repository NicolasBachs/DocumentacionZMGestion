\chapter{Introducción}
\label{ch:capitulo1} 
\markboth{CAPÍTULO \ref*{ch:capitulo1}. Introducción}{CAPÍTULO \ref*{ch:capitulo1}. Introducción}
\section{Introducción}
\paragraph\indent
%El proyecto consiste en realizar el módulo de interfaz gráfica del sistema SCADA desarrollado por DIGICOM.
\paragraph\indent
%El proyecto fue solicitado y financiado por DIGICOM. Actualmente DIGICOM se encuentra en un proceso de modernización de su sistema SCADA el cual está limitado por diferentes características propias de la tecnología disponible en el momento de su desarrollo.
\paragraph\indent
%El sistema SCADA actualmente en funcionamiento fue desarrollado para satisfacer requisitos que a lo largo de los años fueron cambiando.
\paragraph\indent
%El nuevo módulo de interfaz gráfica brindará a los usuarios la capacidad de trabajar con las nuevas características del sistema SCADA, facilitando en gran medida la interacción entre los usuarios y el sistema.

\section{Objetivos}

\subsection{Generales}
\paragraph\indent
\begin{itemize}
	\item Objetivos generales aquí
\end{itemize}


\subsection{Específicos}
\begin{itemize}
	\item Diseñar el sistema de manera que se puedan añadir nuevas funcionalidades al mismo, teniendo en cuenta el constante avance de la tecnología.
	%\item Diseñar vistas cuyas principales funcionalidades puedan ser ejecutadas de manera intuitiva y que al mismo tiempo mantengan un estilo similar a las del sistema que se encuentra actualmente en funcionamiento, haciendo que la capacitación de los usuarios para usar el nuevo sistema sea mucho más sencilla.
\end{itemize}
