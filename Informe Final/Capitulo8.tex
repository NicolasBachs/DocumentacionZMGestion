\chapter{Conclusiones}
\label{ch:capitulo8} 
\markboth{CAPÍTULO \ref*{ch:capitulo8}. Conclusiones}{CAPÍTULO \ref*{ch:capitulo8}. Conclusiones}

\section{Sobre el proyecto}

En primera medida se desarrolló el proyecto cumpliendo con los objetivos propuestos
inicialmente. El sistema desarrollado cumple con los requisitos de software ANSI/IEEE 830. Se pusieron en práctica conocimientos incorporados en la carrera Ingeniería en Computación, y sobre todo en la PPS. También se logró aprender y profundizar acerca de nuevas herramientas necesarias para el desarrollo del proyecto y para una mejor formación profesional.

\paragraph\indent
El sistema permite llevar a cabo todo el proceso que se inicia con la presupuestación y
finaliza con la entrega de los productos, involucrando la venta y fabricación de los mismos. De
esta manera se consiguieron optimizar los procesos que se llevan a cabo en la mueblería, de una manera intuitiva y fácil de utilizar para los empleados.

\paragraph\indent
El sistema permite generar y actualizar la lista de precios de forma automática, reduciendo
los tiempos que tomaba generarlas y distribuirlas en todas las sucursales de la mueblería. Además
otorga la posibilidad de conocer en tiempo real la disponibilidad de los productos en las distintas
sucursales de la empresa. Brindando además información resumida y procesada de las operaciones que se llevan a cabo en la mueblería.

\paragraph\indent
Al usar tecnologías web, se consiguió que el sistema pueda ser ejecutado mediante cualquier
navegador web de los sistemas operativos de mayor uso en la actualidad, Windows, Linux y MacOS.

\paragraph\indent
La implementación del sistema permitirá llevar a cabo un estudio estadístico, mejorando
la toma de decisiones de la mueblería. Además el sistema cuenta con la función de generar
documentos de forma digital, reemplazando los manuscritos, con el objetivo de reducir el impacto
ambiental y el tiempo para su elaboración.

\paragraph\indent
La comunicación con los servidores de datos se encuentra cifrada utilizando protocolos
modernos como ser HTTPS.

\paragraph\indent
En cuanto a las dificultades, una de las principales fue la instalación de los distintos
servidores en entornos Cloud. Para ello se investigó, se tomaron cursos y se realizaron consultas a
expertos en el área. 

\paragraph\indent
Por otro lado, otro reto que se presentó fue el despliegue del sistema utilizando servicios CI/CD automatizado. Para resolver esta problemática se investigó y se realizaron diversas pruebas hasta lograr el objetivo. 

\paragraph\indent
Por último el uso de lenguajes de programación modernos, sin conocimientos previos, obligó a realizar una capacitación en ellos.

% Objetivos:
% Desarrollar un sistema (mediante el uso de tecnologias web) para una mueblería, con el fin de optimizar el proceso de manejo de stock, presupuestos,  ventas y entrega de productos, y que cumpla con las siguientes características:
%   \item permita automatizar los procesos del negocio, SI
%	\item sea intuitivo y fácil de utilizar para los empleados, SI
%	\item se pueda implementar rápidamente, 
%	\item permita almacenar y procesar datos que contribuyan a la toma de decisiones, SI
%	\item sea capaz de brindar información resumida y procesada acerca de las operaciones que se llevan a cabo, SI
%	\item verifique los requisitos de software ANSI/IEEE 830 que se describen en este documento.

\section{Personales}

\paragraph\indent
Desde el punto de vista personal, el desarrollo de este proyecto implicó diversos desafíos que supimos atravesar con éxito. El trabajo en equipo, las largas discusiones sobre diseño, los errores cometidos, los aciertos y el trato con personas son algunos ejemplos.

\paragraph\indent
Con el desarrollo de este proyecto nos enfrentamos a problemas reales, en donde tuvimos que tomar decisiones de diseño ingenieril a fin de encontrar soluciones eficientes que resolviesen los requerimientos del cliente, así como también, otros problemas que se nos presentaron. 

\paragraph\indent
Durante este tiempo mejoramos el trabajo colaborativo, la forma en la que nos relacionamos e interactuamos con el cliente y también supimos diferenciar discusiones relacionadas al proyecto con nuestra relación personal como compañeros.

\paragraph\indent
Tuvimos la oportunidad de hacer nuestra PPS en una empresa que se dedica al desarrollo de software, participando en proyectos cuyos productos tuvieron certificación de calidad internacional. Pudimos aprender los procesos software, la forma de elaborar documentación para certificación, y aprender nuevas herramientas de desarrollo, las cuales fueron aplicadas al proyecto.

\paragraph\indent
Poder ver el sistema de gestión de la mueblería en funcionamiento nos llena de orgullo y nos inspira a seguir trabajando y formándonos en esta hermosa profesión que hemos elegido.