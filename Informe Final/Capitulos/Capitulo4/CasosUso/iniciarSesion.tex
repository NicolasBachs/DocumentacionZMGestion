\renewcommand{\caseUseShortName}{iniciarSesion}
\renewcommand{\caseUseCreated}{17/04/2018}
\renewcommand{\caseUseModified}{17/04/2018}
\renewcommand{\caseUseName}{\CUiniciarSesion - Iniciar sesión}

\renewcommand{\caseUseSummary}{Este caso de uso permite a un usuario iniciar sesión en el sistema.}

\renewcommand{\caseUsePeople}{Usuarios (todos): quiere ingresar al sistema.}

\renewcommand{\caseUsePreconditions}{
	\caseUseRow{No estar logueado en el sistema.}
}

\renewcommand{\caseUsePostconditions}{
	\caseUseRow{Ninguna.}
}

\renewcommand{\caseUseScene}{
	\addCaseUseStep{Un usuario desea ingresar al sistema.}
	\addCaseUseStep{El usuario ingresa nombre de usuario y contraseña.}
	\addCaseUseStep{El usuario hace click en el botón Iniciar sesión.}
	\addCaseUseStep{El sistema envía los datos a la API.} %4
	\addCaseUseStep{La API responde con un código 201 (Éxito).}
	\addCaseUseStep{El sistema autentica al usuario en la conexión websocket.} %6
	\addCaseUseStep{La API responde con un código 201 (Éxito).}
	\addCaseUseStep{El sistema muestra al usuario la pantalla de operación.}
}

\renewcommand{\alternativeCaseUse}{
	\newAlternative{A1: Error al iniciar sesión.}{4}
	\caseUseRow{La secuencia A1 comienza luego del punto 4 del escenario principal.}
	\alternativeRow{La API responde con un código de error 422.}
	\alternativeRow{El sistema muestra al usuario el error ocurrido.}
	\alternativeRow{El sistema borra los datos ingresados por el usuario.}
	\caseUseRow{}
	
	\newAlternative{A2: Error al autenticar al usuario por websocket.}{6}
	\caseUseRow{La secuencia A2 comienza luego del punto 6 del escenario principal.}
	\alternativeRow{La API responde con un código de error 422.}
	\alternativeRow{El sistema vuelve al paso 7 del escenario principal.}
}

\renewcommand{\caseUseRequirementsGUI}{
	\caseUseRow{Teclado, Mouse, Pantalla}
}
\renewcommand{\caseUseResponseTime}{La interfaz debe responder dentro de un tiempo máximo de 5 segundos.}
\renewcommand{\caseUseConcurrence}{}
\renewcommand{\caseUseAvailability}{}
\item Caso de uso: \caseUseName
\renewcommand*{\arraystretch}{1.3}
\begin{longtable}[c]{|>{\raggedright}p{0.3\textwidth} | >{\raggedright}p{0.2\textwidth} | p{0.5\textwidth} |}
\caption{\hyperref[sec:listadoCasoUso]{\caseUseName}}
\label{tabla:\caseUseShortName}\\
\hline
\rowcolor{tableCaseUseBackground}

\multicolumn{3}{|l|}{\textcolor{tableCaseUseFontColor}{Descripción textual del caso de uso: \caseUseName}} \\ \hline

Fecha de Creación: & \multicolumn{2}{L{\secondColumnWidth}|}{\caseUseCreated}\\ \hline

Fecha de Modificación: & \multicolumn{2}{L{\secondColumnWidth}|}{\caseUseModified} \\ \hline

Versión: & \multicolumn{2}{L{\secondColumnWidth}|}{1} \\ \hline

Resumen: & \multicolumn{2}{L{\secondColumnWidth}|}{\caseUseSummary} \\ \hline

Personas involucradas y metas: & \multicolumn{2}{L{\secondColumnWidth}|}{\caseUsePeople} \\ \hline

Precondiciones: \caseUsePreconditions \hline

Postcondiciones: \caseUsePostconditions \hline

Escenario principal: \caseUseScene \hline

Flujos alternativos: \alternativeCaseUse \hline

Requisitos de interfaz de usuario: \caseUseRequirementsGUI \hline
\multirow{3}{*}{Requisitos funcionales:}  & Tiempo de respuesta: & \caseUseResponseTime \\ \cline{2-3} 
& Concurrencia: & \caseUseConcurrence \\ \cline{2-3} 
& Disponibilidad: & \caseUseAvailability \\ \hline
\end{longtable}

\setcounter{rownumbers}{0}

\renewcommand{\alternativeCaseUse}{
	\caseUseRow{No existen flujos alternativos.}
}