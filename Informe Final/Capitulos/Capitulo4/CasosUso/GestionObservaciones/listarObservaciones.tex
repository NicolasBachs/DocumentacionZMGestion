\renewcommand{\caseUseShortName}{listarObservaciones} %cammelCase name
\renewcommand{\caseUseCreated}{05/03/2020} %Fecha creación
\renewcommand{\caseUseModified}{05/03/2020} %Fecha modificación
\renewcommand{\caseUseName}{CU105 - Listar observaciones.} %{\CUcammelCase - Title}

\renewcommand{\caseUseSummary}{Este caso de uso permite a los usaurios de ZMGestion listar las observaciones de una línea de presupuesto, venta, orden de producción o remito.} %Resumen
\renewcommand{\caseUsePeople}{Usuarios: quiere listar las observaciones de una linea de presupuesto, venta, orden de produccióno o remito.} %Actor: Meta
\renewcommand{\caseUsePreconditions}{
    \caseUseRow{Haber realizado con éxito alguno de los siguientes casos de uso: 
        \begin{itemize}
            \item CU71 (Listar líneas de presupuesto).
            \item CU77 (Listar líneas de venta).
            \item CU94 (Listar líneas de orden de producción).
            \item CU51 (Listar líneas de remito.).
        \end{itemize} 
    }
} %Precondiciones

\renewcommand{\caseUsePostconditions}{
	\caseUseRow{Ninguna.} %Postcondiciones
}
\renewcommand{\caseUseScene}{ %Escenario principal
    \addCaseUseStep{El fabricante selecciona una línea de presupuesto, venta orden de producción o remito a la cual le desea listar sus observaciones.}
    \addCaseUseStep{ZMGestion lista las observaciones de la línea seleccionada.}
}
\renewcommand{\alternativeCaseUse}{ %Flujos alternativos
	\newAlternative{A1: La línea seleccionada no posee observaciones.}{1} %Flujo alternativo A1.
	\caseUseRow{La secuencia A1 comienza luego del punto 1 del escenario principal.} %¡Indicar número paso!
    \alternativeRow{ZMGestion muestra un mensaje indicando que la línea seleccionada no posee observaciones.}
    \caseUseRow{El escenario vuelve al punto 1.}
    \caseUseRow{}
}

%\item Caso de uso \caseUseName
\renewcommand*{\arraystretch}{1.3}
\begin{longtable}[c]{|>{\raggedright}p{0.3\textwidth} | >{\raggedright}p{0.2\textwidth} | p{0.5\textwidth} |}
\caption{\hyperref[sec:listadoCasoUso]{\caseUseName}}
\label{tabla:\caseUseShortName}\\
\hline
\rowcolor{tableCaseUseBackground}

\multicolumn{3}{|l|}{\textcolor{tableCaseUseFontColor}{Descripción textual del caso de uso: \caseUseName}} \\ \hline

Fecha de Creación: & \multicolumn{2}{L{\secondColumnWidth}|}{\caseUseCreated}\\ \hline

Fecha de Modificación: & \multicolumn{2}{L{\secondColumnWidth}|}{\caseUseModified} \\ \hline

Versión: & \multicolumn{2}{L{\secondColumnWidth}|}{1} \\ \hline

Resumen: & \multicolumn{2}{L{\secondColumnWidth}|}{\caseUseSummary} \\ \hline

Personas involucradas y metas: & \multicolumn{2}{L{\secondColumnWidth}|}{\caseUsePeople} \\ \hline

Precondiciones: \caseUsePreconditions \hline

Postcondiciones: \caseUsePostconditions \hline

Escenario principal: \caseUseScene \hline

Flujos alternativos: \alternativeCaseUse \hline

Requisitos de interfaz de usuario: \caseUseRequirementsGUI \hline
\multirow{3}{*}{Requisitos funcionales:}  & Tiempo de respuesta: & \caseUseResponseTime \\ \cline{2-3} 
& Concurrencia: & \caseUseConcurrence \\ \cline{2-3} 
& Disponibilidad: & \caseUseAvailability \\ \hline
\end{longtable}

\setcounter{rownumbers}{0}

\renewcommand{\alternativeCaseUse}{
	\caseUseRow{No existen flujos alternativos.}
}

%DIAGRAMA DE ACTIVIDAD
%\lineabreak[0]
%\activityDiagram{\caseUseShortName}{Diagrama de actividad - \caseUseName}