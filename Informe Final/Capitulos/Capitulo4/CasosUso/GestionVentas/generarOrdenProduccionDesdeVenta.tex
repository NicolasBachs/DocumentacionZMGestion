
\renewcommand{\caseUseShortName}{generarOrdenProduccionDesdeVenta} %cammelCase name

\renewcommand{\caseUseCreated}{15/02/2020} %Fecha creación
\renewcommand{\caseUseModified}{15/02/2020} %Fecha modificación
\renewcommand{\caseUseName}{\CUgenerarOrdenProduccionDesdeVenta - Generar orden de producción a partir de venta} %{\CUcammelCase - Title}

\renewcommand{\caseUseSummary}{Este caso de uso permite a un administrador de ZMGestion crear una orden de producción a partir de una venta existente.} %Resumen
\renewcommand{\caseUsePeople}{Administradores: quiere crear una orden de producción a partir de una venta.} %Actor: Meta
\renewcommand{\caseUsePreconditions}{
	\caseUseRow{Haber ejecutado con éxito el \CUbuscarAvanzadoVentas (Buscar avanzado ventas).} %Precondiciones
}
\renewcommand{\caseUsePostconditions}{
	\caseUseRow{Ninguna.} %Postcondiciones
}
\renewcommand{\caseUseScene}{ %Escenario principal
    \addCaseUseStep{El administrador selecciona una venta a partir de la cual desea generar la orden de producción.}
    \addCaseUseStep{ZMGestion ejecuta el \CUlistarLineasVenta\ (Listar líneas de venta) para la venta seleccionada, mostrando una opción por cada línea para indicar si desea añadirla a la orden de producción.}
    \addCaseUseStep{El administrador selecciona las líneas de ventas que desea agregar a la orden de producción.}
    \addCaseUseStep{ZMGestion crea una orden de producción en estado `En creación' y ejecuta el \CUcrearLineaOrdenProduccion\ (Crear línea de orden de producción) con los valores de producto, tela, lustre y cantidad de la línea de venta seleccionada.}
    \addCaseUseStep{Si el administrador desea modificar alguna de las líneas de orden de producción agregadas ejecuta el \CUmodificarLineaOrdenProduccion\ (Modificar línea de orden de producción). Si el administrador desea quitar alguna línea de orden de producción agregada se ejecuta el \CUborrarLineaOrdenProduccion\ (Borrar línea de orden de producción). Si el administrador desea agregar una nueva línea de orden de producción se ejecuta el \CUcrearLineaOrdenProduccion\ (Crear línea de orden de producción).}
    \addCaseUseStep{ZMGestión pasa el estado de la orden de producción a `Pendiente'.}
}
\renewcommand{\alternativeCaseUse}{ %Flujos alternativos
	\newAlternative{A1: No se agregó ninguna línea de orden de producción.}{5} %Flujo alternativo A1.
	\caseUseRow{La secuencia A1 comienza luego del punto 5 del escenario principal.} %¡Indicar número paso!
    \alternativeRow{ZMGestion muestra un mensaje de error indicando que debe ingresar al menos una linea de orden de producción.}
    \caseUseRow{El escenario vuelve al punto 5.}
    \caseUseRow{}
}

\item Caso de uso \caseUseName
\renewcommand*{\arraystretch}{1.3}
\begin{longtable}[c]{|>{\raggedright}p{0.3\textwidth} | >{\raggedright}p{0.2\textwidth} | p{0.5\textwidth} |}
\caption{\hyperref[sec:listadoCasoUso]{\caseUseName}}
\label{tabla:\caseUseShortName}\\
\hline
\rowcolor{tableCaseUseBackground}

\multicolumn{3}{|l|}{\textcolor{tableCaseUseFontColor}{Descripción textual del caso de uso: \caseUseName}} \\ \hline

Fecha de Creación: & \multicolumn{2}{L{\secondColumnWidth}|}{\caseUseCreated}\\ \hline

Fecha de Modificación: & \multicolumn{2}{L{\secondColumnWidth}|}{\caseUseModified} \\ \hline

Versión: & \multicolumn{2}{L{\secondColumnWidth}|}{1} \\ \hline

Resumen: & \multicolumn{2}{L{\secondColumnWidth}|}{\caseUseSummary} \\ \hline

Personas involucradas y metas: & \multicolumn{2}{L{\secondColumnWidth}|}{\caseUsePeople} \\ \hline

Precondiciones: \caseUsePreconditions \hline

Postcondiciones: \caseUsePostconditions \hline

Escenario principal: \caseUseScene \hline

Flujos alternativos: \alternativeCaseUse \hline

Requisitos de interfaz de usuario: \caseUseRequirementsGUI \hline
\multirow{3}{*}{Requisitos funcionales:}  & Tiempo de respuesta: & \caseUseResponseTime \\ \cline{2-3} 
& Concurrencia: & \caseUseConcurrence \\ \cline{2-3} 
& Disponibilidad: & \caseUseAvailability \\ \hline
\end{longtable}

\setcounter{rownumbers}{0}

\renewcommand{\alternativeCaseUse}{
	\caseUseRow{No existen flujos alternativos.}
}

%DIAGRAMA DE ACTIVIDAD
%\lineabreak[0]
%\activityDiagram{\caseUseShortName}{Diagrama de actividad - \caseUseName}