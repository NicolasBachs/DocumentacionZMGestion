
\renewcommand{\caseUseShortName}{modificarVenta} %cammelCase name

\renewcommand{\caseUseCreated}{13/02/2020} %Fecha creación
\renewcommand{\caseUseModified}{13/02/2020} %Fecha modificación
\renewcommand{\caseUseName}{\CUmodificarVenta - Modificar venta} %{\CUcammelCase - Title}

\renewcommand{\caseUseSummary}{Este caso de uso permite a un vendedor modificar una venta existente.} %Resumen
\renewcommand{\caseUsePeople}{Vendedores: quiere modificar una venta existente.} %Actor: Meta
\renewcommand{\caseUsePreconditions}{
	\caseUseRow{Haber realizado con éxito el \CUbuscarAvanzadoVentas\ (Buscar avanzado ventas).} %Precondiciones
}
\renewcommand{\caseUsePostconditions}{
	\caseUseRow{Ninguna.} %Postcondiciones
}
\renewcommand{\caseUseScene}{ %Escenario principal
    \addCaseUseStep{El vendedor indica la venta que desea modificar.}%1
    \addCaseUseStep{ZMGestion muestra un formulario autocompletado para que el usuario modifique el cliente.}%2
    \addCaseUseStep{El vendedor modifica el cliente.}%3
    \addCaseUseStep{Se ejecuta el \CUlistarLineasVenta\ (Listar lineas de venta) para la venta seleccionada.}%4
    \addCaseUseStep{Si el vendedor desea modificar una linea, ejecuta el \CUmodificarLineaVenta\ (Modificar linea de venta), si desea borrar una linea ejecuta el \CUborrarLineaVenta\ (Borrar linea de venta) y si desea agrear una linea ejecuta el \CUcrearLineaVenta\ (Crear linea de venta).}%5
    \addCaseUseStep{ZMGestion la venta y muestra un mensaje indicando el éxito de la operacioón.}
}
\renewcommand{\alternativeCaseUse}{ %Flujos alternativos
	\newAlternative{A1: La venta seleccionada no se encuentra en estado `En creación'.}{1} %Flujo alternativo A1.
	\caseUseRow{La secuencia A1 comienza luego del punto 1 del escenario principal.} %¡Indicar número paso!
    \alternativeRow{ZMGestion muestra un mensaje indicando que no se puede modificar la venta ya que no se encuentra en estado `En creación'.}
    \caseUseRow{El escenario vuelve al punto 1.}
    \caseUseRow{}
    \newAlternative{A2: El vendedor ha dejado el campo de cliente vacío.}{3} %Flujo alternativo A2.
	\caseUseRow{La secuencia A2 comienza luego del punto 3 del escenario principal.} %¡Indicar número paso!
    \alternativeRow{ZMGestion muestra un mensaje indicando que el campo cliente es requerido.}
    \caseUseRow{El escenario vuelve al punto 2.}
    \caseUseRow{}
    \newAlternative{A3: El vendedor que creó la venta no es el mismo que el que quiere modificarla y no es un administrador.}{1} %Flujo alternativo A3.
	\caseUseRow{La secuencia A3 comienza luego del punto 1 del escenario principal.} %¡Indicar número paso!
    \alternativeRow{ZMGestion muestra un mensaje informando que no puede modificar la venta de otro vendedor.}
    \caseUseRow{El escenario vuelve al punto 1.}
    \caseUseRow{}
}
\item Caso de uso \caseUseName
\renewcommand*{\arraystretch}{1.3}
\begin{longtable}[c]{|>{\raggedright}p{0.3\textwidth} | >{\raggedright}p{0.2\textwidth} | p{0.5\textwidth} |}
\caption{\hyperref[sec:listadoCasoUso]{\caseUseName}}
\label{tabla:\caseUseShortName}\\
\hline
\rowcolor{tableCaseUseBackground}

\multicolumn{3}{|l|}{\textcolor{tableCaseUseFontColor}{Descripción textual del caso de uso: \caseUseName}} \\ \hline

Fecha de Creación: & \multicolumn{2}{L{\secondColumnWidth}|}{\caseUseCreated}\\ \hline

Fecha de Modificación: & \multicolumn{2}{L{\secondColumnWidth}|}{\caseUseModified} \\ \hline

Versión: & \multicolumn{2}{L{\secondColumnWidth}|}{1} \\ \hline

Resumen: & \multicolumn{2}{L{\secondColumnWidth}|}{\caseUseSummary} \\ \hline

Personas involucradas y metas: & \multicolumn{2}{L{\secondColumnWidth}|}{\caseUsePeople} \\ \hline

Precondiciones: \caseUsePreconditions \hline

Postcondiciones: \caseUsePostconditions \hline

Escenario principal: \caseUseScene \hline

Flujos alternativos: \alternativeCaseUse \hline

Requisitos de interfaz de usuario: \caseUseRequirementsGUI \hline
\multirow{3}{*}{Requisitos funcionales:}  & Tiempo de respuesta: & \caseUseResponseTime \\ \cline{2-3} 
& Concurrencia: & \caseUseConcurrence \\ \cline{2-3} 
& Disponibilidad: & \caseUseAvailability \\ \hline
\end{longtable}

\setcounter{rownumbers}{0}

\renewcommand{\alternativeCaseUse}{
	\caseUseRow{No existen flujos alternativos.}
}

%DIAGRAMA DE ACTIVIDAD
%\lineabreak[0]
\activityDiagram{\caseUseShortName}{Diagrama de actividad - \caseUseName}