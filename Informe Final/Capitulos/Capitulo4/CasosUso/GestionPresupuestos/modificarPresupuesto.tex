
\renewcommand{\caseUseShortName}{modificarPresupuesto} %cammelCase name

\renewcommand{\caseUseCreated}{04/02/2020} %Fecha creación
\renewcommand{\caseUseModified}{04/02/2020} %Fecha modificación
\renewcommand{\caseUseName}{\CUmodificarPresupuesto - Modificar presupuesto} %{\CUcammelCase - Title}

\renewcommand{\caseUseSummary}{Este caso de uso permite a un vendedor modificar un presupuesto existente.} %Resumen
\renewcommand{\caseUsePeople}{Vendedores: quiere modificar un presupuesto existente.} %Actor: Meta
\renewcommand{\caseUsePreconditions}{
	\caseUseRow{Haber realizado con éxito el \CUbuscarAvanzadoPresupuestos\ (Buscar avanzado presupuestos).} %Precondiciones
}
\renewcommand{\caseUsePostconditions}{
	\caseUseRow{Ninguna.} %Postcondiciones
}
\renewcommand{\caseUseScene}{ %Escenario principal
    \addCaseUseStep{El vendedor indica el presupuesto que desea modificar.}
    \addCaseUseStep{ZMGestion muestra un formulario autocompletado para que el usuario modifique el cliente y/o el periodo de validez. Si el vendedor cuenta con los permisos necesarios podrá modificar el periodo de validez.}
    \addCaseUseStep{El vendedor modifica el cliente y periodo de validez.}
    \addCaseUseStep{Se ejecuta el \CUlistarLineasPresupuesto\ (Listar lineas de presupuesto) para el presupuesto seleccionado.}
    \addCaseUseStep{Si el vendedor desea modificar una linea, ejecuta el \CUmodificarLineaPresupuesto\ (Modificar linea de presupuesto), si desea borrar una linea ejecuta el \CUborrarLineaPresupuesto\ (Borrar linea de presupuesto) y si desea agregar una nueva linea ejecuta el \CUcrearLineaPresupuesto\ (Crear linea de presupuesto).}
    \addCaseUseStep{ZMGestion modifica el presupuesto seleccionado y muestra un mensaje indicando el éxito de la operación.}
}
\renewcommand{\alternativeCaseUse}{ %Flujos alternativos
	\newAlternative{A1: El presupuesto se encuentra en estado `Vendido'.}{5} %Flujo alternativo A1.
	\caseUseRow{La secuencia A1 comienza luego del punto 5 del escenario principal.} %¡Indicar número paso!
    \alternativeRow{ZMGestion muestra un mensaje indicando que no se puede modificar el presupuesto ya que se encuentra en estado "Vendido".}
    \caseUseRow{El escenario vuelve al punto 1.}
    \caseUseRow{}
    \newAlternative{A2: El vendedor modificó el periodo de validez y no cuenta con los permisos necesarios para hacerlo.}{5} %Flujo alternativo A2.
	\caseUseRow{La secuencia A2 comienza luego del punto 5 del escenario principal.} %¡Indicar número paso!
    \alternativeRow{ZMGestion muestra un mensaje indicando que no cuenta con los permisos necesarios para modificar el periodo de validez.}
    \caseUseRow{El escenario vuelve al punto 4.}
    \caseUseRow{}
    \newAlternative{A3: El vendedor ha dejado el campo de cliente o periodo de validez vacío.}{5} %Flujo alternativo A2.
	\caseUseRow{La secuencia A3 comienza luego del punto 5 del escenario principal.} %¡Indicar número paso!
    \alternativeRow{ZMGestion muestra un mensaje indicando que los campos son requeridos.}
    \caseUseRow{El escenario vuelve al punto 4.}
    \caseUseRow{}
    \newAlternative{A4: El vendedor que creó el presupuesto no es el mismo que el que quiere modificarlo y no es un administrador.}{5} %Flujo alternativo A4.
	\caseUseRow{La secuencia A4 comienza luego del punto 5 del escenario principal.} %¡Indicar número paso!
    \alternativeRow{ZMGestion muestra un mensaje informando que no puede modificar el presupuesto de otro vendedor.}
    \caseUseRow{El escenario vuelve al punto 4.}
    \caseUseRow{}
}
\item Caso de uso \caseUseName
\renewcommand*{\arraystretch}{1.3}
\begin{longtable}[c]{|>{\raggedright}p{0.3\textwidth} | >{\raggedright}p{0.2\textwidth} | p{0.5\textwidth} |}
\caption{\hyperref[sec:listadoCasoUso]{\caseUseName}}
\label{tabla:\caseUseShortName}\\
\hline
\rowcolor{tableCaseUseBackground}

\multicolumn{3}{|l|}{\textcolor{tableCaseUseFontColor}{Descripción textual del caso de uso: \caseUseName}} \\ \hline

Fecha de Creación: & \multicolumn{2}{L{\secondColumnWidth}|}{\caseUseCreated}\\ \hline

Fecha de Modificación: & \multicolumn{2}{L{\secondColumnWidth}|}{\caseUseModified} \\ \hline

Versión: & \multicolumn{2}{L{\secondColumnWidth}|}{1} \\ \hline

Resumen: & \multicolumn{2}{L{\secondColumnWidth}|}{\caseUseSummary} \\ \hline

Personas involucradas y metas: & \multicolumn{2}{L{\secondColumnWidth}|}{\caseUsePeople} \\ \hline

Precondiciones: \caseUsePreconditions \hline

Postcondiciones: \caseUsePostconditions \hline

Escenario principal: \caseUseScene \hline

Flujos alternativos: \alternativeCaseUse \hline

Requisitos de interfaz de usuario: \caseUseRequirementsGUI \hline
\multirow{3}{*}{Requisitos funcionales:}  & Tiempo de respuesta: & \caseUseResponseTime \\ \cline{2-3} 
& Concurrencia: & \caseUseConcurrence \\ \cline{2-3} 
& Disponibilidad: & \caseUseAvailability \\ \hline
\end{longtable}

\setcounter{rownumbers}{0}

\renewcommand{\alternativeCaseUse}{
	\caseUseRow{No existen flujos alternativos.}
}

%DIAGRAMA DE ACTIVIDAD
%\lineabreak[0]
\activityDiagram{\caseUseShortName}{Diagrama de actividad - \caseUseName}