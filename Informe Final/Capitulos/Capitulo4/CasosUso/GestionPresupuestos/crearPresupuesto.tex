
\renewcommand{\caseUseShortName}{crearPresupuesto} %cammelCase name

\renewcommand{\caseUseCreated}{04/02/2020} %Fecha creación
\renewcommand{\caseUseModified}{04/02/2020} %Fecha modificación
\renewcommand{\caseUseName}{\CUcrearPresupuesto - Crear presupuesto} %{\CUcammelCase - Title}

\renewcommand{\caseUseSummary}{Este caso de uso permite a un vendedor de ZMGestion crear un presupuesto para un determinado cliente.} %Resumen
\renewcommand{\caseUsePeople}{Vendedores: quiere crear un presupuesto.} %Actor: Meta
\renewcommand{\caseUsePreconditions}{
	\caseUseRow{Haber iniciado sesión en el sistemay tener el permiso necesario para realizar esta función.} %Precondiciones
}
\renewcommand{\caseUsePostconditions}{
	\caseUseRow{Ninguna.} %Postcondiciones
}
\renewcommand{\caseUseScene}{ %Escenario principal
    \addCaseUseStep{El vendedor accede a la pantalla para crear presupuestos.}
    \addCaseUseStep{ZMGestion le muestra un formulario para que el vendedor seleccione un cliente. Además se muestra un campo autocompletado con el periodo de validez. Si el vendedor cuenta con los permisos necesarios para modificar el periodo de validez el campo se muestra habilitado, caso contrario el campo se le muestra deshabilitado.}
    \addCaseUseStep{El vendedor selecciona un cliente y modifica el periodo de validez.}%3
    \addCaseUseStep{ZMGestion crea un presupuesto en estado de "En creación" para el cliente con el periodo de validez ingresado por el vendedor.}%4
    \addCaseUseStep{ZMGestion pregunta si desea agregar una nueva linea de presupuesto.}%5
    \addCaseUseStep{En caso afirmativo se ejecuta el \CUcrearLineaPresupuesto (Crear linea de presupuesto) y se vuelve a realizar la pregunta. En caso contrario ZMGestión pasa el estado del presupuesto a "Creado".}
    \addCaseUseStep{ZMGestión muestra un mensaje indicando el éxito de la operacion y vuelve al punto 5 del escenario principal.
    }
}
\renewcommand{\alternativeCaseUse}{ %Flujos alternativos
	\newAlternative{A1: No ha seleccionado ningún cliente.}{3} %Flujo alternativo A1.
	\caseUseRow{La secuencia A1 comienza luego del punto 3 del escenario principal.} %¡Indicar número paso!
    \alternativeRow{ZMGestion muestra un mensaje de error indicando que debe seleccionar un cliente.}
    \caseUseRow{El escenario vuelve al punto 2.}
    \caseUseRow{}
	\newAlternative{A2: No ha agregado ninguna linea de presupuesto.}{6} %Flujo alternativo A2.
    \caseUseRow{La secuencia A2 comienza luego del punto 6 del escenario principal.}%¡Indicar número paso!
    \alternativeRow{ZMGestion muestra un mensaje de error indicando que debe agregar al menos una linea de presupuesto.}
    \caseUseRow{El escenario vuelve al punto 5.}
    \caseUseRow{}
}

\item Caso de uso \caseUseName
\renewcommand*{\arraystretch}{1.3}
\begin{longtable}[c]{|>{\raggedright}p{0.3\textwidth} | >{\raggedright}p{0.2\textwidth} | p{0.5\textwidth} |}
\caption{\hyperref[sec:listadoCasoUso]{\caseUseName}}
\label{tabla:\caseUseShortName}\\
\hline
\rowcolor{tableCaseUseBackground}

\multicolumn{3}{|l|}{\textcolor{tableCaseUseFontColor}{Descripción textual del caso de uso: \caseUseName}} \\ \hline

Fecha de Creación: & \multicolumn{2}{L{\secondColumnWidth}|}{\caseUseCreated}\\ \hline

Fecha de Modificación: & \multicolumn{2}{L{\secondColumnWidth}|}{\caseUseModified} \\ \hline

Versión: & \multicolumn{2}{L{\secondColumnWidth}|}{1} \\ \hline

Resumen: & \multicolumn{2}{L{\secondColumnWidth}|}{\caseUseSummary} \\ \hline

Personas involucradas y metas: & \multicolumn{2}{L{\secondColumnWidth}|}{\caseUsePeople} \\ \hline

Precondiciones: \caseUsePreconditions \hline

Postcondiciones: \caseUsePostconditions \hline

Escenario principal: \caseUseScene \hline

Flujos alternativos: \alternativeCaseUse \hline

Requisitos de interfaz de usuario: \caseUseRequirementsGUI \hline
\multirow{3}{*}{Requisitos funcionales:}  & Tiempo de respuesta: & \caseUseResponseTime \\ \cline{2-3} 
& Concurrencia: & \caseUseConcurrence \\ \cline{2-3} 
& Disponibilidad: & \caseUseAvailability \\ \hline
\end{longtable}

\setcounter{rownumbers}{0}

\renewcommand{\alternativeCaseUse}{
	\caseUseRow{No existen flujos alternativos.}
}

%DIAGRAMA DE ACTIVIDAD
%\lineabreak[0]
%\activityDiagram{AD_\caseUseShortName}{Diagrama de actividad - \caseUseName}