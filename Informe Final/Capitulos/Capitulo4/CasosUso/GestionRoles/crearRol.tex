
\renewcommand{\caseUseShortName}{crearRol} %cammelCase name

\renewcommand{\caseUseCreated}{31/01/2020} %Fecha creación
\renewcommand{\caseUseModified}{31/01/2020} %Fecha modificación
\renewcommand{\caseUseName}{\CUcrearRol - Crear rol } %{\CUcammelCase - Title}

\renewcommand{\caseUseSummary}{Este caso de uso permite a un administrador de ZMGestion crear un nuevo rol.} %Resumen
\renewcommand{\caseUsePeople}{Administrador: quiere crear un nuevo rol.} %Actor: Meta
\renewcommand{\caseUsePreconditions}{
	\caseUseRow{Haber iniciado sesión en el sistema y tener el permiso necesario para realizar esta función.} %Precondiciones
}
\renewcommand{\caseUsePostconditions}{
	\caseUseRow{Ninguna.} %Postcondiciones
}
\renewcommand{\caseUseScene}{ %Escenario principal
    \addCaseUseStep{El administrador accede a la pantalla para crear roles.}
    \addCaseUseStep{ZMGestion muestra un formulario para que el administrador ingrese el nombre del rol y seleccione todos los permisos que desea otorgarle. Indicando que ambos campos son obligatorios.}
    \addCaseUseStep{El administrador completa los campos requeridos del formulario.}
    \addCaseUseStep{ZMGestion crea el rol y muestra un mensaje indicando el éxito de la operación.}
}
\renewcommand{\alternativeCaseUse}{ %Flujos alternativos
	\newAlternative{A1: El nombre del rol ingresado ya esta en uso.}{3} %Flujo alternativo A1.
	\caseUseRow{La secuencia A1 comienza luego del punto 3 del escenario principal.} %¡Indicar número paso!
    \alternativeRow{ZMgestion muestra un mensaje indicando que el nombre ingresado ya se encuentra en uso.}
    \caseUseRow{El escenario vuelve al punto 2.}    
    \caseUseRow{}

	\newAlternative{A2:El administrador ha dejado un campo obligatorio vacio.}{3} %Flujo alternativo A2.
    \caseUseRow{La secuencia A2 comienza luego del punto 3 del escenario principal.}%¡Indicar número paso!
    \alternativeRow{ZMGestion muestra un mensaje de error indicando que dicho campo es requerido.}
    \caseUseRow{EL escenario vuelve al punto 2.}
    \caseUseRow{}
}

\item Caso de uso \caseUseName
\renewcommand*{\arraystretch}{1.3}
\begin{longtable}[c]{|>{\raggedright}p{0.3\textwidth} | >{\raggedright}p{0.2\textwidth} | p{0.5\textwidth} |}
\caption{\hyperref[sec:listadoCasoUso]{\caseUseName}}
\label{tabla:\caseUseShortName}\\
\hline
\rowcolor{tableCaseUseBackground}

\multicolumn{3}{|l|}{\textcolor{tableCaseUseFontColor}{Descripción textual del caso de uso: \caseUseName}} \\ \hline

Fecha de Creación: & \multicolumn{2}{L{\secondColumnWidth}|}{\caseUseCreated}\\ \hline

Fecha de Modificación: & \multicolumn{2}{L{\secondColumnWidth}|}{\caseUseModified} \\ \hline

Versión: & \multicolumn{2}{L{\secondColumnWidth}|}{1} \\ \hline

Resumen: & \multicolumn{2}{L{\secondColumnWidth}|}{\caseUseSummary} \\ \hline

Personas involucradas y metas: & \multicolumn{2}{L{\secondColumnWidth}|}{\caseUsePeople} \\ \hline

Precondiciones: \caseUsePreconditions \hline

Postcondiciones: \caseUsePostconditions \hline

Escenario principal: \caseUseScene \hline

Flujos alternativos: \alternativeCaseUse \hline

Requisitos de interfaz de usuario: \caseUseRequirementsGUI \hline
\multirow{3}{*}{Requisitos funcionales:}  & Tiempo de respuesta: & \caseUseResponseTime \\ \cline{2-3} 
& Concurrencia: & \caseUseConcurrence \\ \cline{2-3} 
& Disponibilidad: & \caseUseAvailability \\ \hline
\end{longtable}

\setcounter{rownumbers}{0}

\renewcommand{\alternativeCaseUse}{
	\caseUseRow{No existen flujos alternativos.}
}

%DIAGRAMA DE ACTIVIDAD
%\lineabreak[0]
%\activityDiagram{AD_\caseUseShortName}{Diagrama de actividad - \caseUseName}