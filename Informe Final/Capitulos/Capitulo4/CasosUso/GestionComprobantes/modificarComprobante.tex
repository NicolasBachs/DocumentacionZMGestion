\renewcommand{\caseUseShortName}{modificarComprobante} %cammelCase name

\renewcommand{\caseUseCreated}{07/03/2020} %Fecha creación
\renewcommand{\caseUseModified}{07/03/2020} %Fecha modificación
\renewcommand{\caseUseName}{\CUmodificarComprobante - Modificar comprobante} %{\CUcammelCase - Title}

\renewcommand{\caseUseSummary}{Este caso de uso permite a un administrador modificar un comprobante existente.} %Resumen
\renewcommand{\caseUsePeople}{Administradores: quiere modificar un comprobante existente.} %Actor: Meta
\renewcommand{\caseUsePreconditions}{
	\caseUseRow{Haber realizado con éxito el \CUbuscarAvanzadoComprobantes\ (Buscar avanzado comprobantes).} %Precondiciones
}
\renewcommand{\caseUsePostconditions}{
	\caseUseRow{Ninguna.} %Postcondiciones
}
\renewcommand{\caseUseScene}{ %Escenario principal
    \addCaseUseStep{El administrador indica el comprobante que desea modificar.}%1
    \addCaseUseStep{ZMGestion muestra un formulario autocompletado con el número de comprobante, tipo de comprobante, monto total y observaciones. Indicando que todos los campos son obligatorios excepto el de obserevaciones.}%2
    \addCaseUseStep{El vendedor modifica el formulario.}
    \addCaseUseStep{ZMGestion modifica el comprobante con los nuevos valores ingresados por el administrador.}
}
\renewcommand{\alternativeCaseUse}{ %Flujos alternativos
	\newAlternative{A1: El número y tipo de comprobante ingresado ya existe.}{3} %Flujo alternativo A1.
	\caseUseRow{La secuencia A1 comienza luego del punto 3 del escenario principal.} %¡Indicar número paso!
    \alternativeRow{ZMGestion muestra un mensaje indicando que el comprobante ingresado ya existe.}
    \caseUseRow{El escenario vuelve al punto 2.}
    \caseUseRow{}
    \newAlternative{A2: El monto ingresado es menor o igual a cero.}{3} %Flujo alternativo A2.
	\caseUseRow{La secuencia A2 comienza luego del punto 3 del escenario principal.} %¡Indicar número paso!
    \alternativeRow{ZMGestion muestra un mensaje indicando que debe ingresar un monto válido.}
    \caseUseRow{El escenario vuelve al punto 2.}
    \caseUseRow{}
    \newAlternative{A3: Se ha dejado un campo requerido vacío.}{3} %Flujo alternativo A3.
	\caseUseRow{La secuencia A3 comienza luego del punto 3 del escenario principal.} %¡Indicar número paso!
    \alternativeRow{ZMGestion muestra un mensaje indicando que debe completar todos los campos requeridos.}
    \caseUseRow{El escenario vuelve al punto 2.}
    \caseUseRow{}
}
%\item Caso de uso \caseUseName
\renewcommand*{\arraystretch}{1.3}
\begin{longtable}[c]{|>{\raggedright}p{0.3\textwidth} | >{\raggedright}p{0.2\textwidth} | p{0.5\textwidth} |}
\caption{\hyperref[sec:listadoCasoUso]{\caseUseName}}
\label{tabla:\caseUseShortName}\\
\hline
\rowcolor{tableCaseUseBackground}

\multicolumn{3}{|l|}{\textcolor{tableCaseUseFontColor}{Descripción textual del caso de uso: \caseUseName}} \\ \hline

Fecha de Creación: & \multicolumn{2}{L{\secondColumnWidth}|}{\caseUseCreated}\\ \hline

Fecha de Modificación: & \multicolumn{2}{L{\secondColumnWidth}|}{\caseUseModified} \\ \hline

Versión: & \multicolumn{2}{L{\secondColumnWidth}|}{1} \\ \hline

Resumen: & \multicolumn{2}{L{\secondColumnWidth}|}{\caseUseSummary} \\ \hline

Personas involucradas y metas: & \multicolumn{2}{L{\secondColumnWidth}|}{\caseUsePeople} \\ \hline

Precondiciones: \caseUsePreconditions \hline

Postcondiciones: \caseUsePostconditions \hline

Escenario principal: \caseUseScene \hline

Flujos alternativos: \alternativeCaseUse \hline

Requisitos de interfaz de usuario: \caseUseRequirementsGUI \hline
\multirow{3}{*}{Requisitos funcionales:}  & Tiempo de respuesta: & \caseUseResponseTime \\ \cline{2-3} 
& Concurrencia: & \caseUseConcurrence \\ \cline{2-3} 
& Disponibilidad: & \caseUseAvailability \\ \hline
\end{longtable}

\setcounter{rownumbers}{0}

\renewcommand{\alternativeCaseUse}{
	\caseUseRow{No existen flujos alternativos.}
}

%DIAGRAMA DE ACTIVIDAD
%\lineabreak[0]
%\activityDiagram{\caseUseShortName}{Diagrama de actividad - \caseUseName}