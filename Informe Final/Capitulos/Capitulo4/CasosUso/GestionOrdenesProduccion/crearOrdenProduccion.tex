\renewcommand{\caseUseShortName}{crearOrdenProduccion} %cammelCase name

\renewcommand{\caseUseCreated}{02/03/2020} %Fecha creación
\renewcommand{\caseUseModified}{02/03/2020} %Fecha modificación
\renewcommand{\caseUseName}{\CUcrearOrdenProduccion - Crear orden de producción} %{\CUcammelCase - Title}

\renewcommand{\caseUseSummary}{Este caso de uso permite a un administrador de ZMGestion crear una orden de producción.} %Resumen
\renewcommand{\caseUsePeople}{Administradores: quiere crear una orden de producción.} %Actor: Meta
\renewcommand{\caseUsePreconditions}{
	\caseUseRow{Haber iniciado sesión en el sistema y tener el permiso necesario para realizar esta función.} %Precondiciones
}
\renewcommand{\caseUsePostconditions}{
	\caseUseRow{Ninguna.} %Postcondiciones
}
\renewcommand{\caseUseScene}{ %Escenario principal
    \addCaseUseStep{El administrador accede a la pantalla para crear órdenes de producción.}
    \addCaseUseStep{ZMGestion crea una orden de producción en estado de `En creación'.}%4
    \addCaseUseStep{Si el administrador desea agregar una linea de orden de producción ejecuta el \CUcrearLineaOrdenProduccion (Crear linea de orden de producción). Si el vendedor desea borrar una linea de orden de producción se ejecuta el \CUborrarLineaOrdenProduccion (Borrar linea de orden de producción). Si el administrador desea modificar una linea de orden de producción se ejecuta el \CUmodificarLineaOrdenProduccion (Modificar linea de orden de producción).}%5
    \addCaseUseStep{ZMGestion pasa el estado de la orden de producción a `Pendiente'.}
}
\renewcommand{\alternativeCaseUse}{ %Flujos alternativos
	\newAlternative{A1: No se agregó ninguna linea de orden de producción.}{3} %Flujo alternativo A1.
	\caseUseRow{La secuencia A1 comienza luego del punto 3 del escenario principal.} %¡Indicar número paso!
    \alternativeRow{ZMGestion muestra un mensaje de error indicando que no se pueden agregar dos lineas idénticas a la orden de producción.}
    \caseUseRow{El escenario vuelve al punto 3.}
    \caseUseRow{}
}

%\item Caso de uso \caseUseName
\renewcommand*{\arraystretch}{1.3}
\begin{longtable}[c]{|>{\raggedright}p{0.3\textwidth} | >{\raggedright}p{0.2\textwidth} | p{0.5\textwidth} |}
\caption{\hyperref[sec:listadoCasoUso]{\caseUseName}}
\label{tabla:\caseUseShortName}\\
\hline
\rowcolor{tableCaseUseBackground}

\multicolumn{3}{|l|}{\textcolor{tableCaseUseFontColor}{Descripción textual del caso de uso: \caseUseName}} \\ \hline

Fecha de Creación: & \multicolumn{2}{L{\secondColumnWidth}|}{\caseUseCreated}\\ \hline

Fecha de Modificación: & \multicolumn{2}{L{\secondColumnWidth}|}{\caseUseModified} \\ \hline

Versión: & \multicolumn{2}{L{\secondColumnWidth}|}{1} \\ \hline

Resumen: & \multicolumn{2}{L{\secondColumnWidth}|}{\caseUseSummary} \\ \hline

Personas involucradas y metas: & \multicolumn{2}{L{\secondColumnWidth}|}{\caseUsePeople} \\ \hline

Precondiciones: \caseUsePreconditions \hline

Postcondiciones: \caseUsePostconditions \hline

Escenario principal: \caseUseScene \hline

Flujos alternativos: \alternativeCaseUse \hline

Requisitos de interfaz de usuario: \caseUseRequirementsGUI \hline
\multirow{3}{*}{Requisitos funcionales:}  & Tiempo de respuesta: & \caseUseResponseTime \\ \cline{2-3} 
& Concurrencia: & \caseUseConcurrence \\ \cline{2-3} 
& Disponibilidad: & \caseUseAvailability \\ \hline
\end{longtable}

\setcounter{rownumbers}{0}

\renewcommand{\alternativeCaseUse}{
	\caseUseRow{No existen flujos alternativos.}
}

%DIAGRAMA DE ACTIVIDAD
%\lineabreak[0]
\activityDiagram{\caseUseShortName}{Diagrama de actividad - \caseUseName}