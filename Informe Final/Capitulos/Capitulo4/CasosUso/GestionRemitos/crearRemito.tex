
\renewcommand{\caseUseShortName}{crearRemito} %cammelCase name

\renewcommand{\caseUseCreated}{05/03/2020} %Fecha creación
\renewcommand{\caseUseModified}{05/03/2020} %Fecha modificación
\renewcommand{\caseUseName}{\CUcrearRemito - Crear remito } %{\CUcammelCase - Title}

\renewcommand{\caseUseSummary}{Este caso de uso permite a un vendedor de ZMGestion crear un remito.} %Resumen
\renewcommand{\caseUsePeople}{Vendedores: quiere crear un remito.} %Actor: Meta
\renewcommand{\caseUsePreconditions}{
	\caseUseRow{Haber iniciado sesión en el sistema y tener el permiso necesario para realizar esta función.} %Precondiciones
}
\renewcommand{\caseUsePostconditions}{
	\caseUseRow{Ninguna.} %Postcondiciones
}
\renewcommand{\caseUseScene}{ %Escenario principal
    \addCaseUseStep{El vendedor accede a la pantalla para crear remitos.}
    \addCaseUseStep{ZMGestion muestra un formulario para que el vendedor seleccione el tipo de remito (Entrada o Salida), la ubicación, de entrada o salida según corresponda e ingrese observaciones,. Indicando que todos los campos son obligatorios excepto por las observaciones.}
    \addCaseUseStep{El vendedor completa los campos requeridos.} 
    \addCaseUseStep{Si el vendedor desea agregar una linea de remito se ejecuta el \CUcrearLineaRemito (Crear linea de remito). Si el vendedor desea borrar una linea de remito se ejecuta el \CUborrarLineaRemito (Borrar linea de remito). Si el vendedor desea modificar una linea de remito se ejecuta el \CUmodificarLineaRemito (Modificar linea de remito).}
    \addCaseUseStep{ZMGestion crea el remito con los valores indicados por el vendedor.}
}
\renewcommand{\alternativeCaseUse}{ %Flujos alternativos
	\newAlternative{A1: El vendedor ha dejado un campo obligatorio vacío.}{2} %Flujo alternativo A1.
	\caseUseRow{La secuencia A1 comienza luego del punto 2 del escenario principal.} %¡Indicar número paso!
    \alternativeRow{ZMGestion muestra un mensaje de error indicando que ha dejado un campo obligatorio vacío.}
    \caseUseRow{El escenario vuelve al punto 2.}
    \caseUseRow{}
    
    \newAlternative{A2: No se ha agregado ningúna línea de remito.}{4} %Flujo alternativo A1.
	\caseUseRow{La secuencia A1 comienza luego del punto 4 del escenario principal.} %¡Indicar número paso!
    \alternativeRow{ZMGestion muestra un mensaje de error indicando que debe agregar al menos una línea de remito.}
    \caseUseRow{El escenario vuelve al punto 4.}
    \caseUseRow{}
}

\item Caso de uso \caseUseName
\renewcommand*{\arraystretch}{1.3}
\begin{longtable}[c]{|>{\raggedright}p{0.3\textwidth} | >{\raggedright}p{0.2\textwidth} | p{0.5\textwidth} |}
\caption{\hyperref[sec:listadoCasoUso]{\caseUseName}}
\label{tabla:\caseUseShortName}\\
\hline
\rowcolor{tableCaseUseBackground}

\multicolumn{3}{|l|}{\textcolor{tableCaseUseFontColor}{Descripción textual del caso de uso: \caseUseName}} \\ \hline

Fecha de Creación: & \multicolumn{2}{L{\secondColumnWidth}|}{\caseUseCreated}\\ \hline

Fecha de Modificación: & \multicolumn{2}{L{\secondColumnWidth}|}{\caseUseModified} \\ \hline

Versión: & \multicolumn{2}{L{\secondColumnWidth}|}{1} \\ \hline

Resumen: & \multicolumn{2}{L{\secondColumnWidth}|}{\caseUseSummary} \\ \hline

Personas involucradas y metas: & \multicolumn{2}{L{\secondColumnWidth}|}{\caseUsePeople} \\ \hline

Precondiciones: \caseUsePreconditions \hline

Postcondiciones: \caseUsePostconditions \hline

Escenario principal: \caseUseScene \hline

Flujos alternativos: \alternativeCaseUse \hline

Requisitos de interfaz de usuario: \caseUseRequirementsGUI \hline
\multirow{3}{*}{Requisitos funcionales:}  & Tiempo de respuesta: & \caseUseResponseTime \\ \cline{2-3} 
& Concurrencia: & \caseUseConcurrence \\ \cline{2-3} 
& Disponibilidad: & \caseUseAvailability \\ \hline
\end{longtable}

\setcounter{rownumbers}{0}

\renewcommand{\alternativeCaseUse}{
	\caseUseRow{No existen flujos alternativos.}
}

%DIAGRAMA DE ACTIVIDAD
%\lineabreak[0]
%\activityDiagram{\caseUseShortName}{Diagrama de actividad - \caseUseName}