
\renewcommand{\caseUseShortName}{modificarLineaVenta} %cammelCase name

\renewcommand{\caseUseCreated}{13/02/2020} %Fecha creación
\renewcommand{\caseUseModified}{13/02/2020} %Fecha modificación
\renewcommand{\caseUseName}{CU85\ - Modificar linea de venta} %{\CUcammelCase - Title}

\renewcommand{\caseUseSummary}{Este caso de uso permite a un vendedor modificar una linea de venta existente.} %Resumen
\renewcommand{\caseUsePeople}{Vendedores: quiere modificar una linea de venta.} %Actor: Meta
\renewcommand{\caseUsePreconditions}{
	\caseUseRow{Haber ejecutado con éxito el \CUlistarLineasVenta (Listar líneas de venta)} %Precondiciones
}
\renewcommand{\caseUsePostconditions}{
	\caseUseRow{Si el producto, tela y lustre seleccionado no pertenece a ningún producto final existente se ejecuta el CU25 (Crear producto final) con el producto, tela y lustre seleccionados.} %Postcondiciones
}
\renewcommand{\caseUseScene}{ %Escenario principal
    \addCaseUseStep{El vendedor indica la linea de venta que desea modificar.}
    \addCaseUseStep{ZMGestion le muestra un formulario autocompletado para que el vendedor modifique el producto, tela, lustre, precio unitario y la cantidad solicitada de dicho producto. En caso de no contar con los permisos necesarios para asignar precios el campo de precio unitario se muestra deshabilitado.}
    \addCaseUseStep{El vendedor modifica el producto, tela, lustre y la cantidad solicitada.}
    \addCaseUseStep{ZMGestion autocompleta el campo precio unitario con el precio actual del producto seleccionado con la tela solicitada. Si el vendedor cuenta con los permisos necesario para modificar este campo se le permite la edición, caso contrario se le muestra el campo deshabilitado con el precio actual del producto con la tela seleccionada.}
    \addCaseUseStep{El vendedor cuenta con los permisos necesarios y modifica el precio.}
    \addCaseUseStep{ZMGestion modifica la linea de venta con el valor de los campos producto, tela, lustre, precio unitario y cantidad solicitada ingresados por el vendedor.}
}
\renewcommand{\alternativeCaseUse}{ %Flujos alternativos
	\newAlternative{A1: La venta a la cual pertenece la linea no se encuentra en estado `En creación'.}{3} %Flujo alternativo A1.
	\caseUseRow{La secuencia A1 comienza luego del punto 3 del escenario principal.} %¡Indicar número paso!
    \alternativeRow{ZMGestion muestra un mensaje de error indicando que no se puede modificar la linea de venta ya que la venta no se encuentra en estado `En creación'.}
    \caseUseRow{El escenario vuelve al punto 2.}
    \caseUseRow{}

    \newAlternative{A2: La cantidad solicitada es menor o igual a cero.}{3} %Flujo alternativo A2.
	\caseUseRow{La secuencia A2 comienza luego del punto 3 del escenario principal.} %¡Indicar número paso!
    \alternativeRow{ZMGestion muestra un mensaje de error indicando que debe ingresar una cantidad mayor a cero.}
    \caseUseRow{El escenario vuelve al punto 2.}
    \caseUseRow{}

    \newAlternative{A3: El vendedor no cuenta con los permisos necesarios para modificar el precio.}{5} %Flujo alternativo A3.
    \caseUseRow{La secuencia A3 comienza luego del punto 5 del escenario principal.}%¡Indicar número paso!
    \alternativeRow{ZMGestion muestra un mensaje de error indicando que no cuenta con los permisos necesarios para modificar el precio unitario.}
    \caseUseRow{El escenario vuelve al punto 4.}
}

%\item Caso de uso \caseUseName
\renewcommand*{\arraystretch}{1.3}
\begin{longtable}[c]{|>{\raggedright}p{0.3\textwidth} | >{\raggedright}p{0.2\textwidth} | p{0.5\textwidth} |}
\caption{\hyperref[sec:listadoCasoUso]{\caseUseName}}
\label{tabla:\caseUseShortName}\\
\hline
\rowcolor{tableCaseUseBackground}

\multicolumn{3}{|l|}{\textcolor{tableCaseUseFontColor}{Descripción textual del caso de uso: \caseUseName}} \\ \hline

Fecha de Creación: & \multicolumn{2}{L{\secondColumnWidth}|}{\caseUseCreated}\\ \hline

Fecha de Modificación: & \multicolumn{2}{L{\secondColumnWidth}|}{\caseUseModified} \\ \hline

Versión: & \multicolumn{2}{L{\secondColumnWidth}|}{1} \\ \hline

Resumen: & \multicolumn{2}{L{\secondColumnWidth}|}{\caseUseSummary} \\ \hline

Personas involucradas y metas: & \multicolumn{2}{L{\secondColumnWidth}|}{\caseUsePeople} \\ \hline

Precondiciones: \caseUsePreconditions \hline

Postcondiciones: \caseUsePostconditions \hline

Escenario principal: \caseUseScene \hline

Flujos alternativos: \alternativeCaseUse \hline

Requisitos de interfaz de usuario: \caseUseRequirementsGUI \hline
\multirow{3}{*}{Requisitos funcionales:}  & Tiempo de respuesta: & \caseUseResponseTime \\ \cline{2-3} 
& Concurrencia: & \caseUseConcurrence \\ \cline{2-3} 
& Disponibilidad: & \caseUseAvailability \\ \hline
\end{longtable}

\setcounter{rownumbers}{0}

\renewcommand{\alternativeCaseUse}{
	\caseUseRow{No existen flujos alternativos.}
}

%DIAGRAMA DE ACTIVIDAD
%\lineabreak[0]
%\activityDiagram{\caseUseShortName}{Diagrama de actividad - \caseUseName}