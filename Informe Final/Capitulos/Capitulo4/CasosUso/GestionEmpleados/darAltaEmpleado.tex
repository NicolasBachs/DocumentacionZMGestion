
\renewcommand{\caseUseShortName}{darAltaEmpleado} %cammelCase name

\renewcommand{\caseUseCreated}{27/01/2020} %Fecha creación
\renewcommand{\caseUseModified}{27/01/2020} %Fecha modificación
\renewcommand{\caseUseName}{\CUdarAltaEmpleado - Dar de alta empleado} %{\CUcammelCase - Title}

\renewcommand{\caseUseSummary}{Este caso de uso permite a un administrador dar de alta un empleado que se encuentra en el estado de Baja.} %Resumen
\renewcommand{\caseUsePeople}{Administradores: quiere dar de alta un empleado que se encuentra en estado Baja.} %Actor: Meta
\renewcommand{\caseUsePreconditions}{
	\caseUseRow{Haber realizado con éxito el \CUbuscarAvanzadoEmpleados (Buscar avanzado empleados).} %Precondiciones
}
\renewcommand{\caseUsePostconditions}{
	\caseUseRow{Ningúna.} %Postcondiciones
}
\renewcommand{\caseUseScene}{ %Escenario principal
    \addCaseUseStep{El administrador indica el usuario que desea dar de alta.}
    \addCaseUseStep{ZMGestión da de alta el usuario indicado y muestra un mensaje informando que la operación se realizó con éxito.}
}
\renewcommand{\alternativeCaseUse}{ %Flujos alternativos
	\newAlternative{A1: El usuario ya se encuentra en estado Activo.}{1} %Flujo alternativo A1.
	\caseUseRow{La secuencia A1 comienza luego del punto 1 del escenario principal.} %¡Indicar número paso!
    \alternativeRow{ZMGestion informa que el usuario indicado ya que encuentra en estado Activo.}
    \caseUseRow{El escenario vuelve al punto 1.}
}

\item Caso de uso \caseUseName
\renewcommand*{\arraystretch}{1.3}
\begin{longtable}[c]{|>{\raggedright}p{0.3\textwidth} | >{\raggedright}p{0.2\textwidth} | p{0.5\textwidth} |}
\caption{\hyperref[sec:listadoCasoUso]{\caseUseName}}
\label{tabla:\caseUseShortName}\\
\hline
\rowcolor{tableCaseUseBackground}

\multicolumn{3}{|l|}{\textcolor{tableCaseUseFontColor}{Descripción textual del caso de uso: \caseUseName}} \\ \hline

Fecha de Creación: & \multicolumn{2}{L{\secondColumnWidth}|}{\caseUseCreated}\\ \hline

Fecha de Modificación: & \multicolumn{2}{L{\secondColumnWidth}|}{\caseUseModified} \\ \hline

Versión: & \multicolumn{2}{L{\secondColumnWidth}|}{1} \\ \hline

Resumen: & \multicolumn{2}{L{\secondColumnWidth}|}{\caseUseSummary} \\ \hline

Personas involucradas y metas: & \multicolumn{2}{L{\secondColumnWidth}|}{\caseUsePeople} \\ \hline

Precondiciones: \caseUsePreconditions \hline

Postcondiciones: \caseUsePostconditions \hline

Escenario principal: \caseUseScene \hline

Flujos alternativos: \alternativeCaseUse \hline

Requisitos de interfaz de usuario: \caseUseRequirementsGUI \hline
\multirow{3}{*}{Requisitos funcionales:}  & Tiempo de respuesta: & \caseUseResponseTime \\ \cline{2-3} 
& Concurrencia: & \caseUseConcurrence \\ \cline{2-3} 
& Disponibilidad: & \caseUseAvailability \\ \hline
\end{longtable}

\setcounter{rownumbers}{0}

\renewcommand{\alternativeCaseUse}{
	\caseUseRow{No existen flujos alternativos.}
}

%DIAGRAMA DE ACTIVIDAD
%\lineabreak[0]
%\activityDiagram{AD_\caseUseShortName}{Diagrama de actividad - \caseUseName}