
\renewcommand{\caseUseShortName}{modificarEmpleado} %cammelCase name

\renewcommand{\caseUseCreated}{27/01/2020} %Fecha creación
\renewcommand{\caseUseModified}{27/01/2020} %Fecha modificación
\renewcommand{\caseUseName}{\CUmodificarEmpleado - Modificar empleado} %{\CUcammelCase - Title}

\renewcommand{\caseUseSummary}{Este caso de uso permite a un administrador de ZMGestion modificar un empleado existente.} %Resumen
\renewcommand{\caseUsePeople}{Administradores: quiere modificar un empleado.} %Actor: Meta
\renewcommand{\caseUsePreconditions}{
	\caseUseRow{Haber realizado con éxito el \CUbuscarAvanzadoEmpleados (Buscar avanzado empleados).} %Precondiciones
}
\renewcommand{\caseUsePostconditions}{
	\caseUseRow{Ninguna.} %Postcondiciones
}
\renewcommand{\caseUseScene}{ %Escenario principal
    \addCaseUseStep{El administrador indica el usuario que desea modificar.}
    \addCaseUseStep{ZMGestion muestra un formulario autocompletado con los datos del usuario seleccionado para que el administrador modifique: Nombres, apellidos, correo electrónico, número de teléfono, nombre de usuario, contraseña, fecha de inicio de actividad laboral, cantidad de hijos, estado civil, tipo de documento, documento, teléfono, fecha de nacimiento y/o rol del empleado. Indicando que son requeridos todos los campos.}
    \addCaseUseStep{El usuario modifica los campos que desea cambiar.}
    \addCaseUseStep{ZMGestion modifica el usuario con los nuevos valores de los campos solicitados.}
}
\renewcommand{\alternativeCaseUse}{ %Flujos alternativos
	\newAlternative{A1: El nombre de usuario ingresado está siendo usado por otro empleado.}{3} %Flujo alternativo A1.
	\caseUseRow{La secuencia A1 comienza luego del punto 3 del escenario principal.} %¡Indicar número paso!
    \alternativeRow{ZMGestion muestra un mensaje de error informando que el nombre de usuario ingresado ya está en uso.}
    \caseUseRow{El escenario vuelve al punto 2.}
    \caseUseRow{}
    \newAlternative{A2: El correo electrónico ingresado está siendo usado por otro empleado.}{3} %Flujo alternativo A2.
	\caseUseRow{La secuencia A2 comienza luego del punto 3 del escenario principal.} %¡Indicar número paso!
    \alternativeRow{ZMGestion muestra un mensaje de error informando que el correo electrónico ingresado ya está en uso.}
    \caseUseRow{El escenario vuelve al punto 2.}
    \caseUseRow{}
    \newAlternative{A3: El documento y tipo de documento ingresado está siendo usado por otro empleado.}{3} %Flujo alternativo A3.
	\caseUseRow{La secuencia A3 comienza luego del punto 3 del escenario principal.} %¡Indicar número paso!
    \alternativeRow{ZMGestion muestra un mensaje de error informando que el documento y tipo de documento ya existe.}
    \caseUseRow{El escenario vuelve al punto 2.}
    \caseUseRow{}
    \newAlternative{A4: El usuario ha modificado el rol del usuario.}{3} %Flujo alternativo A3.
	\caseUseRow{La secuencia A4 comienza luego del punto 3 del escenario principal.} %¡Indicar número paso!
    \alternativeRow{ZMGestion cierra la sesión del usuario que se está modificando.}
    \caseUseRow{El escenario continúa desde el punto 4.}
    \caseUseRow{}
    \newAlternative{A5: El usuario ha dejado un campo requerido vacio.}{3} %Flujo alternativo A3.
	\caseUseRow{La secuencia A5 comienza luego del punto 3 del escenario principal.} %¡Indicar número paso!
    \alternativeRow{ZMGestion informa al usuario que dicho campo es requerido.}
    \caseUseRow{El escenario vuelve al punto 2.}
    \caseUseRow{}
}

\item Caso de uso \caseUseName
\renewcommand*{\arraystretch}{1.3}
\begin{longtable}[c]{|>{\raggedright}p{0.3\textwidth} | >{\raggedright}p{0.2\textwidth} | p{0.5\textwidth} |}
\caption{\hyperref[sec:listadoCasoUso]{\caseUseName}}
\label{tabla:\caseUseShortName}\\
\hline
\rowcolor{tableCaseUseBackground}

\multicolumn{3}{|l|}{\textcolor{tableCaseUseFontColor}{Descripción textual del caso de uso: \caseUseName}} \\ \hline

Fecha de Creación: & \multicolumn{2}{L{\secondColumnWidth}|}{\caseUseCreated}\\ \hline

Fecha de Modificación: & \multicolumn{2}{L{\secondColumnWidth}|}{\caseUseModified} \\ \hline

Versión: & \multicolumn{2}{L{\secondColumnWidth}|}{1} \\ \hline

Resumen: & \multicolumn{2}{L{\secondColumnWidth}|}{\caseUseSummary} \\ \hline

Personas involucradas y metas: & \multicolumn{2}{L{\secondColumnWidth}|}{\caseUsePeople} \\ \hline

Precondiciones: \caseUsePreconditions \hline

Postcondiciones: \caseUsePostconditions \hline

Escenario principal: \caseUseScene \hline

Flujos alternativos: \alternativeCaseUse \hline

Requisitos de interfaz de usuario: \caseUseRequirementsGUI \hline
\multirow{3}{*}{Requisitos funcionales:}  & Tiempo de respuesta: & \caseUseResponseTime \\ \cline{2-3} 
& Concurrencia: & \caseUseConcurrence \\ \cline{2-3} 
& Disponibilidad: & \caseUseAvailability \\ \hline
\end{longtable}

\setcounter{rownumbers}{0}

\renewcommand{\alternativeCaseUse}{
	\caseUseRow{No existen flujos alternativos.}
}

%DIAGRAMA DE ACTIVIDAD
%\lineabreak[0]
%\activityDiagram{AD_\caseUseShortName}{Diagrama de actividad - \caseUseName}