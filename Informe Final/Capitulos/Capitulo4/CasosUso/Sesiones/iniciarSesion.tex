\renewcommand{\caseUseShortName}{iniciarSesion}
\renewcommand{\caseUseCreated}{16/01/2020}
\renewcommand{\caseUseModified}{16/01/2020}
\renewcommand{\caseUseName}{\CUiniciarSesion - Iniciar sesión}
\renewcommand{\caseUseSummary}{Este caso de uso permite a un usuario iniciar sesión en el sistema.}
\renewcommand{\caseUsePeople}{Usuarios: quiere ingresar al sistema.}
\renewcommand{\caseUsePreconditions}{
	\caseUseRow{El usuario se encuentra creado y activo en ZMGestion.}
}
\renewcommand{\caseUsePostconditions}{
	\caseUseRow{Se valida al usuario, se genera el token de sesión y se le muestra al usuario las opciones personales disponibles.}
}
\renewcommand{\caseUseScene}{
	\addCaseUseStep{El usuario ingresa la dirección de la aplicación en un dispositivo conectado a Internet.}
	\addCaseUseStep{ZMGestion muestra un formulario para que el usuario ingrese su nombre de usuario y contraseña.}
	\addCaseUseStep{El usuario introduce su nombre de usuario y contraseña.}
	\addCaseUseStep{ZMGestion genera el token correspondiente y lo registra en su sesión.} %4
	\addCaseUseStep{ZMGestion trae los permisos del usuario y le muestra sus opciones.}
	
}
\renewcommand{\alternativeCaseUse}{
	\newAlternative{A1: El nombre de usuario no existe en ZMGestion.}{3}
	\caseUseRow{La secuencia A1 comienza luego del punto 3 del escenario principal.}
	\alternativeRow{ZMGestion muestra un mensaje de error.}
	\caseUseRow{El escenario vuelve al punto 2.}
	\caseUseRow{}
	
	\newAlternative{A2: El usuario no se encuentra activo.}{3}
	\caseUseRow{La secuencia A2 comienza luego del punto 3 del escenario principal.}
	\alternativeRow{ZMGestion informa al usuario que el mismo no se encuentra activo y que debe comuicarse con un administrador.}
	\caseUseRow{El escenario vuelve al punto 2.}
	\caseUseRow{}

	\newAlternative{A3: La contraseña ingresada es incorrecta y el número de intentos no supero el limite permitido.}{3}
	\caseUseRow{La secuencia A3 comienza luego del punto 3 del escenario principal.}
	\alternativeRow{ZMGestion informa al usuario que la contraseña ingresada es incorrecta.}
	\caseUseRow{El escenario vuelve al punto 2.}
	\caseUseRow{}

	\newAlternative{A4: La contraseña ingresada es incorrecta y el número de intentos supero el limite permitido.}{3}
	\caseUseRow{La secuencia A4 comienza luego del punto 3 del escenario principal.}
	\alternativeRow{ZMGestion informa al usuario que la contraseña es incorrecta, que el usuario ha sido bloqueado y que debe comunicarse con un administrador.}
	\caseUseRow{El escenario vuelve al punto 2.}
	\caseUseRow{}	
}

\item Caso de uso \caseUseName

%DESCRIPCION TEXTUAL
\renewcommand*{\arraystretch}{1.3}
\begin{longtable}[c]{|>{\raggedright}p{0.3\textwidth} | >{\raggedright}p{0.2\textwidth} | p{0.5\textwidth} |}
\caption{\hyperref[sec:listadoCasoUso]{\caseUseName}}
\label{tabla:\caseUseShortName}\\
\hline
\rowcolor{tableCaseUseBackground}

\multicolumn{3}{|l|}{\textcolor{tableCaseUseFontColor}{Descripción textual del caso de uso: \caseUseName}} \\ \hline

Fecha de Creación: & \multicolumn{2}{L{\secondColumnWidth}|}{\caseUseCreated}\\ \hline

Fecha de Modificación: & \multicolumn{2}{L{\secondColumnWidth}|}{\caseUseModified} \\ \hline

Versión: & \multicolumn{2}{L{\secondColumnWidth}|}{1} \\ \hline

Resumen: & \multicolumn{2}{L{\secondColumnWidth}|}{\caseUseSummary} \\ \hline

Personas involucradas y metas: & \multicolumn{2}{L{\secondColumnWidth}|}{\caseUsePeople} \\ \hline

Precondiciones: \caseUsePreconditions \hline

Postcondiciones: \caseUsePostconditions \hline

Escenario principal: \caseUseScene \hline

Flujos alternativos: \alternativeCaseUse \hline

Requisitos de interfaz de usuario: \caseUseRequirementsGUI \hline
\multirow{3}{*}{Requisitos funcionales:}  & Tiempo de respuesta: & \caseUseResponseTime \\ \cline{2-3} 
& Concurrencia: & \caseUseConcurrence \\ \cline{2-3} 
& Disponibilidad: & \caseUseAvailability \\ \hline
\end{longtable}

\setcounter{rownumbers}{0}

\renewcommand{\alternativeCaseUse}{
	\caseUseRow{No existen flujos alternativos.}
}

%DIAGRAMA DE ACTIVIDAD
%\lineabreak[0]
\activityDiagram{\caseUseShortName}{Diagrama de actividad - \caseUseName}
