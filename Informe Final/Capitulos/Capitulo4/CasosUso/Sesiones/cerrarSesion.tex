\renewcommand{\caseUseShortName}{cerrarSesion}
\renewcommand{\caseUseCreated}{16/01/2020}
\renewcommand{\caseUseModified}{16/01/2020}
\renewcommand{\caseUseName}{CU02 - Cerrar sesión}
\renewcommand{\caseUseSummary}{Este caso de uso permite a un usuario cerrar sesión en el sistema.}
\renewcommand{\caseUsePeople}{Usuarios: quiere salir del sistema.}
\renewcommand{\caseUsePreconditions}{
	\caseUseRow{Tener una sesión iniciada en el sistema.}
}
\renewcommand{\caseUsePostconditions}{
	\caseUseRow{Se borra el token de sesión almacenado en el dispositivo del usuario.}
}
\renewcommand{\caseUseScene}{
	\addCaseUseStep{El usuario accede a la pantalla destinada para cerrar sesión.}
	\addCaseUseStep{ZMGestion cierra la sesión del usuario y muestra un mensaje informando el éxito de la operación.}
}
\renewcommand{\alternativeCaseUse}{
	\caseUseRow{Ninguno.}
	\caseUseRow{}
}

%\item Caso de uso \caseUseName

%DESCRIPCION TEXTUAL
\renewcommand*{\arraystretch}{1.3}
\begin{longtable}[c]{|>{\raggedright}p{0.3\textwidth} | >{\raggedright}p{0.2\textwidth} | p{0.5\textwidth} |}
\caption{\hyperref[sec:listadoCasoUso]{\caseUseName}}
\label{tabla:\caseUseShortName}\\
\hline
\rowcolor{tableCaseUseBackground}

\multicolumn{3}{|l|}{\textcolor{tableCaseUseFontColor}{Descripción textual del caso de uso: \caseUseName}} \\ \hline

Fecha de Creación: & \multicolumn{2}{L{\secondColumnWidth}|}{\caseUseCreated}\\ \hline

Fecha de Modificación: & \multicolumn{2}{L{\secondColumnWidth}|}{\caseUseModified} \\ \hline

Versión: & \multicolumn{2}{L{\secondColumnWidth}|}{1} \\ \hline

Resumen: & \multicolumn{2}{L{\secondColumnWidth}|}{\caseUseSummary} \\ \hline

Personas involucradas y metas: & \multicolumn{2}{L{\secondColumnWidth}|}{\caseUsePeople} \\ \hline

Precondiciones: \caseUsePreconditions \hline

Postcondiciones: \caseUsePostconditions \hline

Escenario principal: \caseUseScene \hline

Flujos alternativos: \alternativeCaseUse \hline

Requisitos de interfaz de usuario: \caseUseRequirementsGUI \hline
\multirow{3}{*}{Requisitos funcionales:}  & Tiempo de respuesta: & \caseUseResponseTime \\ \cline{2-3} 
& Concurrencia: & \caseUseConcurrence \\ \cline{2-3} 
& Disponibilidad: & \caseUseAvailability \\ \hline
\end{longtable}

\setcounter{rownumbers}{0}

\renewcommand{\alternativeCaseUse}{
	\caseUseRow{No existen flujos alternativos.}
}

%DIAGRAMA DE ACTIVIDAD
%\lineabreak[0]
%\activityDiagram{\caseUseShortName}{Diagrama de actividad - \caseUseName}
