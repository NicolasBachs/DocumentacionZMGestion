
\renewcommand{\caseUseShortName}{crearProducto} %cammelCase name

\renewcommand{\caseUseCreated}{27/01/2020} %Fecha creación
\renewcommand{\caseUseModified}{27/01/2020} %Fecha modificación
\renewcommand{\caseUseName}{CU13 - Crear producto } %{\CUcammelCase - Title}

\renewcommand{\caseUseSummary}{Este caso de uso permite a los administradores de ZMGestion crear un producto de manera segura y confiable.} %Resumen
\renewcommand{\caseUsePeople}{Administradores: quiere crear un producto.} %Actor: Meta
\renewcommand{\caseUsePreconditions}{
	\caseUseRow{Haber iniciado sesión en el sistema y tener el permiso necesario para realizar esta función.} %Precondiciones
}
\renewcommand{\caseUsePostconditions}{
	\caseUseRow{Ninguna.} %Postcondiciones
}
\renewcommand{\caseUseScene}{ %Escenario principal
    \addCaseUseStep{El administrador accede a la pantalla para crear productos. }
    \addCaseUseStep{ZMGestion muestra un formulario para que el administrador ingrese el nombre del producto, el precio unitario, la cantidad de metros de tela que deben utilizarse para producirlo, tipo de producto, observaciones y dos listas para seleccionar el grupo y categoría de productos a los cuales pertenece el mismo. Indicando que todos los campos menos la cantidad de metros de tela y observaciones son obligatorios.}
    \addCaseUseStep{El administrador completa los campos del formulario.}
    \addCaseUseStep{ZMGestion crea al producto con los campos ingresados por el usuario y muestra un mensaje indicando el éxito de la operación. }
}
\renewcommand{\alternativeCaseUse}{ %Flujos alternativos
	\newAlternative{A1: El nombre del producto ingresado, la categoría y grupo seleccionado ya existe.}{3} %Flujo alternativo A1.
	\caseUseRow{La secuencia A1 comienza luego del punto 3 del escenario principal.} %¡Indicar número paso!
    \alternativeRow{ZMGestion muestra un mensaje de error indicando que el nombre del producto ya está en uso para el grupo y categoría seleccionado.}
    \caseUseRow{El escenario vuelve al punto 2.}
    \caseUseRow{}

    \newAlternative{A2: El administrador ha dejado un campo obligatorio vacío.}{3} %Flujo alternativo A1.
	\caseUseRow{La secuencia A2 comienza luego del punto 3 del escenario principal.} %¡Indicar número paso!
    \alternativeRow{ZMGestion muestra un mensaje de error indicando que dicho campo es requerido.}
    \caseUseRow{El escenario vuelve al punto 2.}
    \caseUseRow{}

    \newAlternative{A3: El precio ingresado es menor o igual a cero.}{3} %Flujo alternativo A1.
	\caseUseRow{La secuencia A3 comienza luego del punto 3 del escenario principal.} %¡Indicar número paso!
    \alternativeRow{ZMGestion muestra un mensaje de error informando que el precio no puede ser menor o igual que cero<.}
    \caseUseRow{El escenario vuelve al punto 2.}
    \caseUseRow{}

    \newAlternative{A4: La cantidad de tela requerida ingresada es menor a cero.}{3} %Flujo alternativo A1.
	\caseUseRow{La secuencia A4 comienza luego del punto 3 del escenario principal.} %¡Indicar número paso!
    \alternativeRow{ZMGestion muestra un mensaje de error informando que la cantidad de tela necesaria ingresada es inválida.}
    \caseUseRow{El escenario vuelve al punto 2.}
    \caseUseRow{}

}

%\item Caso de uso \caseUseName
\renewcommand*{\arraystretch}{1.3}
\begin{longtable}[c]{|>{\raggedright}p{0.3\textwidth} | >{\raggedright}p{0.2\textwidth} | p{0.5\textwidth} |}
\caption{\hyperref[sec:listadoCasoUso]{\caseUseName}}
\label{tabla:\caseUseShortName}\\
\hline
\rowcolor{tableCaseUseBackground}

\multicolumn{3}{|l|}{\textcolor{tableCaseUseFontColor}{Descripción textual del caso de uso: \caseUseName}} \\ \hline

Fecha de Creación: & \multicolumn{2}{L{\secondColumnWidth}|}{\caseUseCreated}\\ \hline

Fecha de Modificación: & \multicolumn{2}{L{\secondColumnWidth}|}{\caseUseModified} \\ \hline

Versión: & \multicolumn{2}{L{\secondColumnWidth}|}{1} \\ \hline

Resumen: & \multicolumn{2}{L{\secondColumnWidth}|}{\caseUseSummary} \\ \hline

Personas involucradas y metas: & \multicolumn{2}{L{\secondColumnWidth}|}{\caseUsePeople} \\ \hline

Precondiciones: \caseUsePreconditions \hline

Postcondiciones: \caseUsePostconditions \hline

Escenario principal: \caseUseScene \hline

Flujos alternativos: \alternativeCaseUse \hline

Requisitos de interfaz de usuario: \caseUseRequirementsGUI \hline
\multirow{3}{*}{Requisitos funcionales:}  & Tiempo de respuesta: & \caseUseResponseTime \\ \cline{2-3} 
& Concurrencia: & \caseUseConcurrence \\ \cline{2-3} 
& Disponibilidad: & \caseUseAvailability \\ \hline
\end{longtable}

\setcounter{rownumbers}{0}

\renewcommand{\alternativeCaseUse}{
	\caseUseRow{No existen flujos alternativos.}
}

%DIAGRAMA DE ACTIVIDAD
%\lineabreak[0]
\activityDiagram{\caseUseShortName}{Diagrama de actividad - \caseUseName}