\renewcommand{\caseUseShortName}{crearLineaOrdenProduccion} %cammelCase name

\renewcommand{\caseUseCreated}{02/03/2020} %Fecha creación
\renewcommand{\caseUseModified}{02/03/2020} %Fecha modificación
\renewcommand{\caseUseName}{\CUcrearLineaOrdenProduccion\ - Crear linea de orden de producción.} %{\CUcammelCase - Title}

\renewcommand{\caseUseSummary}{Este caso de uso permite a un administrador crear una linea de orden de producción para una determinada orden de producción.} %Resumen
\renewcommand{\caseUsePeople}{Administradores: quiere crear una linea de orden de producción.} %Actor: Meta
\renewcommand{\caseUsePreconditions}{
	\caseUseRow{Estar ejecutando el \CUcrearOrdenProduccion (Crear orden de producción) o el \CUmodificarOrdenProduccion (Modificar orden de producción).} %Precondiciones
}
\renewcommand{\caseUsePostconditions}{
	\caseUseRow{Si el producto, tela y lustre no pertenece a ningun producto final existente se ejecuta el \CUcrearProductoFinal\ (Crear producto final) con el producto, tela y lustre seleccionados.} %Postcondiciones
}
\renewcommand{\caseUseScene}{ %Escenario principal
    \addCaseUseStep{El administrador desea agregar una linea de orden de producción a una determinada orden de producción.}
    \addCaseUseStep{ZMGestion le muestra un formulario para que el administrador seleccione un producto, tela, lustre e indique la cantidad a producir de dicho producto. Indicando que el producto y la cantidad son obligatorios.}
    \addCaseUseStep{El administrador completa el formulario.}
    \addCaseUseStep{ZMGestion crea la linea de orden de producción en estado `Pendiente de producción' y la asocia a la orden de producción correspondiente.}
}
\renewcommand{\alternativeCaseUse}{ %Flujos alternativos
    \newAlternative{A1: La cantidad indicada es menor o igual a cero.}{3} %Flujo alternativo A1.
    \caseUseRow{La secuencia A1 comienza luego del punto 3 del escenario principal.} %¡Indicar número paso!
    \alternativeRow{ZMGestion muestra un mensaje de error indicando que debe ingresar una cantidad mayor a cero.}
    \caseUseRow{El escenario vuelve al punto 2.}
    \caseUseRow{}
    \newAlternative{A2: El producto, tela y lustre indicado ya se encuentra en la orden de producción.}{3} %Flujo alternativo A2.
    \caseUseRow{La secuencia A2 comienza luego del punto 3 del escenario principal.} %¡Indicar número paso!
    \alternativeRow{ZMGestion muestra un mensaje de error indicando que la combinación de producto, tela y lustre ingresado ya se encuentra en la orden de producción.}
    \caseUseRow{El escenario vuelve al punto 2.}
    \caseUseRow{}
    \newAlternative{A3: El tipo de producto seleccionado no es del tipo `Producible'.}{3} %Flujo alternativo A1.
    \caseUseRow{La secuencia A3 comienza luego del punto 3 del escenario principal.} %¡Indicar número paso!
    \alternativeRow{ZMGestion muestra un mensaje de error indicando que el producto seleccionado no puede ser asignado a una orden de producción.}
    \caseUseRow{El escenario vuelve al punto 2.}
    \caseUseRow{}

    \newAlternative{A4: El administrador ha dejado un campo obligatorio vacío.}{3} %Flujo alternativo A1.
    \caseUseRow{La secuencia A4 comienza luego del punto 3 del escenario principal.} %¡Indicar número paso!
    \alternativeRow{ZMGestion muestra un mensaje de error indicando que ha dejado un campo obligatorio vacío.}
    \caseUseRow{El escenario vuelve al punto 2.}
    \caseUseRow{}
}

\item Caso de uso \caseUseName
\renewcommand*{\arraystretch}{1.3}
\begin{longtable}[c]{|>{\raggedright}p{0.3\textwidth} | >{\raggedright}p{0.2\textwidth} | p{0.5\textwidth} |}
\caption{\hyperref[sec:listadoCasoUso]{\caseUseName}}
\label{tabla:\caseUseShortName}\\
\hline
\rowcolor{tableCaseUseBackground}

\multicolumn{3}{|l|}{\textcolor{tableCaseUseFontColor}{Descripción textual del caso de uso: \caseUseName}} \\ \hline

Fecha de Creación: & \multicolumn{2}{L{\secondColumnWidth}|}{\caseUseCreated}\\ \hline

Fecha de Modificación: & \multicolumn{2}{L{\secondColumnWidth}|}{\caseUseModified} \\ \hline

Versión: & \multicolumn{2}{L{\secondColumnWidth}|}{1} \\ \hline

Resumen: & \multicolumn{2}{L{\secondColumnWidth}|}{\caseUseSummary} \\ \hline

Personas involucradas y metas: & \multicolumn{2}{L{\secondColumnWidth}|}{\caseUsePeople} \\ \hline

Precondiciones: \caseUsePreconditions \hline

Postcondiciones: \caseUsePostconditions \hline

Escenario principal: \caseUseScene \hline

Flujos alternativos: \alternativeCaseUse \hline

Requisitos de interfaz de usuario: \caseUseRequirementsGUI \hline
\multirow{3}{*}{Requisitos funcionales:}  & Tiempo de respuesta: & \caseUseResponseTime \\ \cline{2-3} 
& Concurrencia: & \caseUseConcurrence \\ \cline{2-3} 
& Disponibilidad: & \caseUseAvailability \\ \hline
\end{longtable}

\setcounter{rownumbers}{0}

\renewcommand{\alternativeCaseUse}{
	\caseUseRow{No existen flujos alternativos.}
}

%DIAGRAMA DE ACTIVIDAD
%\lineabreak[0]
\activityDiagram{\caseUseShortName}{Diagrama de actividad - \caseUseName}