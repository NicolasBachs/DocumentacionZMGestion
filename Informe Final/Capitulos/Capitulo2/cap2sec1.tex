\section{Análisis de requisitos del sistema}
	\paragraph\indent
	Esta especificación tiene como objetivo analizar y documentar las necesidades funcionales que deberán ser soportadas por el sistema a desarrollar. Para ello, se identificarán los requisitos que ha de satisfacer el nuevo sistema mediante entrevistas, el estudio de los problemas de las unidades afectadas y sus necesidades actuales. Además de identificar los requisitos se deberán establecer prioridades, lo cual proporciona un punto de referencia para validar el sistema final que compruebe que se ajusta a las necesidades del usuario. 
	\subsection{Identificación de los usuarios participantes}
		\paragraph\indent
			Los objetivos de esta tarea son identificar a los responsables de cada una de las unidades implicadas y a los principales usuarios implicados. En la organización se identificaron los siguientes usuarios:
		\begin{itemize}
			\item Gerente de Empresa Zimmerman Muebles SRL: es el solicitante de la aplicación.
			\item Vendedores: Formado por los usuarios capaces de realizar funciones del sistema relacionadas con presupuestos y ventas.
            \item Fabricantes: Formado por aquellos usuarios que llevan a cabo la producción de muebles.
            \item Administradores: Formado por aquellos usuarios que poseen los permisos para realizar todas las funciones del sistema.
		\end{itemize}
		\paragraph\indent
			Es de destacar la necesidad de una participación activa de los usuarios del futuro sistema en las actividades de desarrollo del mismo, con objeto de conseguir la máxima adecuación del sistema a sus necesidades y facilitar el conocimiento paulatino de dicho sistema, permitiendo una rápida implantación.
	\subsection{Catálogo de requisitos del sistema}
		\paragraph\indent
        El objetivo de la especificación es definir en forma clara, precisa, completa y verificable todas las funcionalidades y restricciones del sistema que se desea construir. Esta documentación está sujeta a revisiones por el grupo de usuarios que se recogerán por medio de sucesivas versiones del documento, hasta alcanzar su aprobación por parte de la dirección de Zimmerman Muebles SRL. y del grupo de usuarios. Una vez aprobado, servirá de base al equipo para la construcción del nuevo sistema. 
        \paragraph\indent
			Esta especificación se ha realizado de acuerdo al estándar ``IEEE Recommended Practice for Software Requirements Specifications (IEEE/ANSI 830-1998)''.
		\subsubsection{Objetivos y alcance del sistema}
			\paragraph\indent
            El principal objetivo es desarrollar una aplicación web para el uso exclusivo del personal interno de la mueblería Zimmerman Muebles SRL, a partir de ahora ZM, con el fin de automatizar el proceso de manejo de stock, desde la emisión de presupuestos, pasando por las ventas, hasta la entrega de productos.
			\paragraph\indent
            Se podrá realizar la gestión de usuarios, clientes, productos, presupuestos, ventas, órdenes de producción, remitos y órdenes de reposición. Además, se podrá gestionar la entrega de los productos. El futuro sistema llevará el nombre de \textbf{ZMGestion}.
            \paragraph\indent
            En esta versión se realizará lo recién mencionado dejando para futuras versiones la  contabilidad y la gestión con los proveedores.
            \paragraph\indent
            El desarrollo lo llevará a cabo \textbf{NL}, con opción a ser responsable del posterior mantenimiento de éste.
        \subsubsection{Definiciones, acrónimos y abreviaturas}
            \underline{Definiciones:}
            \begin{itemize}
                \item Interfaz: en este contexto, llamamos interfaz a la pantalla que una persona puede acceder para recibir o transmitir información.
                \item Cliente: toda aquella persona que solicitó un presupuesto y/o efectuó una compra.
                \item ZM: refiere a la empresa Zimmerman Muebles SRL.
            \end{itemize}
            \underline{Acrónimos:}
            \begin{itemize}
                \item ZMGestion: refiere al sistema a desarrollar.
            \end{itemize}
            \underline{Abreviaturas:}
            \begin{itemize}
                \item IEEE: Institute of Electrical and Electronics Engineers.
                \item CUIT: Clave Única de Identificación Tributaria.
                \item CUIL: Clave Única de Identificación Laboral.
            \end{itemize}
            
		\subsubsection{Descripción general}
			\paragraph\indent
				Esta sección presenta una descripción general del sistema con el fin de conocer las funciones que debe soportar, los datos asociados, las restricciones impuestas y cualquier otro factor que pueda influir en la construcción del mismo.
				\paragraph\indent\textbf{Usuarios:}
				\paragraph\indent
				Para poder acceder a ZMGestion se debe contar con una cuenta (nombre de usuario y contraseña). Cada empleado de la mueblería tiene una sola cuenta de usuario, de ellos se desea saber: nombres, apellidos, tipo y número de documento, fecha de nacimiento, fecha de inicio de actividad laboral en la empresa, estado civil, cantidad de hijos, número de teléfono de contacto, correo electrónico y contraseña. Todos los datos recién mencionados son de carácter obligatorio. Todas las funciones del sistema requieren que los usuarios inicien sesión. Los usuarios tienen dos estados posibles: alta o baja. Cuando el estado sea alta, tienen la posibilidad de iniciar sesión y ejercer pleno uso de sus funciones. Mientras que cuando estén dados de baja no pueden iniciar sesión.
				\clearpage
				\paragraph\indent\textbf{Roles:}
				\paragraph\indent	
				Cada empleado tiene un rol, pudiendo ser: Administrador, Vendedor o Productor. Dependiendo del rol que ocupen en la mueblería, cuentan distintos permisos lo que les permite realizar distintas funciones del sistema. Los roles se pueden gestionar (crear, modificar, borrar, dar de alta y dar de baja) por aquellos usuarios que tengan el permiso de hacerlo.
				\paragraph\indent\textbf{Empleados:}
				\paragraph\indent	
				Existe una gestión de empleados (crear, modificar, borrar, dar de alta y dar de baja) que es realizada por los usuarios con los debidos permisos. No se pueden borrar aquellos usuarios que hayan realizado algún presupuesto, venta u orden de producción. O bien, tengan asociada alguna orden de producción.
				\paragraph\indent\textbf{Clientes:}
				\paragraph\indent	
				Un cliente es aquella persona que solicita un presupuesto o realiza una compra. Para que un cliente sea dado de alta es necesario saber: el tipo de persona (física o jurídica), nombres y apellidos (o razón social), tipo y número de documento, correo electrónico, nacionalidad, número de teléfono y domicilio. Siendo obligatorio: el tipo de persona, nombres y apellidos (o razón social), tipo y número de documento y correo electrónico. El correo electrónico es único entre los usuarios y se pueden dar de alta clientes que tengan el mismo tipo y número de documento, emitiendo una advertencia que ya existe un cliente con los mismos datos. Los clientes se pueden gestionar (crear, modificar, borrar, dar de alta y dar de baja) por aquellos usuarios que cuenten con los permisos necesarios. No se puede borrar un cliente que tenga asociado un presupuesto o venta.
				\paragraph\indent\textbf{Presupuestos:}
				\paragraph\indent	
				Al momento de presentarse un cliente solicitando un presupuesto se comprueba si está activo para ello se pregunta el tipo y número de documento y se verifica el resto de los datos. En caso de no existir se procede a crearlo. Un presupuesto tiene un código de identificación, fecha en la que se realiza, periodo de validez, cliente asociado, una o más líneas de presupuesto y estado. Los presupuestos se pueden gestionar (crear, modificar y borrar) por aquellos usuarios que cuenten con los permisos necesarios. No se pueden borrar los presupuestos que hayan sido utilizados para realizar una venta. Los presupuestos son enviados por correo electrónico siempre y cuando el cliente haya proporcionado dicha información.
				\paragraph\indent\textbf{Líneas de presupuesto:}
				\paragraph\indent	
				Una línea de presupuesto está formada por un producto, la cantidad solicitada de dicho producto, el importe unitario, el importe total para dicha cantidad y un estado (pudiendo ser: pendiente, utilizada o no utilizada). Los posibles estados de una línea de presupuesto son:
				\begin{itemize}
					\item \textbf{Pendiente:} Cuando se ha creado una línea de presupuesto.
					\item \textbf{Utilizada:} Cuando la línea de presupuesto ha sido utilizada para crear una venta.
					\item \textbf{No utilizada:} Cuando la línea de presupuesto ha sido descartada al momento de crear una venta a partir del presupuesto al cual pertenece.
					\item \textbf{Expirado:} Cuando la línea de presupuesto ha superado la cantidad de días en la cual es válido.
				\end{itemize}
				Las líneas de presupuesto se pueden gestionar (crear, modificar y borrar) por aquellos usuarios que cuenten con los permisos necesarios. Además algunos usuarios cuentan con un permiso que les permite modificar el importe unitario o total de una línea de presupuesto.
				\paragraph\indent\textbf{Productos:}
				\paragraph\indent	
				Un producto tiene nombre, código identificador (único para cada producto), tipo de producto ( con proceso de producción, sin proceso de producción o a medida), lustre y tela. Además cada producto pertenece a una categoría y grupo. Todos los datos de los productos son obligatorios excepto la tela y/o lustre para aquellos sin proceso de producción. Los usuarios que cuenten con los permisos necesarios pueden gestionar productos (crear, modificar, borrar, dar de alta y dar de baja). No se pueden borrar productos que hayan sido utilizados en algún presupuesto, venta, orden de producción u orden de reposición.
				\paragraph\indent\textbf{Categorías de producto:}
				\paragraph\indent	
				Los productos pertenecen a una única categoría y una categoría puede tener cero o más productos. De la categoría se desea saber el nombre y estado (pudiendo ser alta o baja). Las categorías se pueden gestionar (crear, modificar, borrar, dar de alta y dar de baja) por aquellos usuarios que cuenten con  los permisos suficientes. No se pueden borrar categorías que tengan al menos un producto asociado.
				\paragraph\indent\textbf{Grupos de productos:}
				\paragraph\indent	
				Los productos se encuentran asociados en grupos de productos. Estos están compuestos por cero o más productos. De los grupos se desea saber: código, estado (pudiendo ser alta o baja) y descripción. Los grupos de productos se pueden gestionar (crear, modificar, borrar, dar de alta y dar de baja) por aquellos usuarios que cuenten con  los permisos suficientes. No se pueden borrar grupos que estén compuestos por al menos un producto.
				\paragraph\indent\textbf{Ubicaciones y Stock:}
				\paragraph\indent	
				Los productos tienen asociado una o más ubicaciones geográficas. De las ubicaciones se desea saber: nombre y dirección. Las ubicaciones pueden tener almacenados cero o más productos. Las ubicaciones se pueden gestionar (crear, modificar, borrar, dar de alta y dar de baja) por aquellos usuarios que cuenten con  los permisos suficientes.. Por cada producto se desea saber: cantidad por ubicación a una fecha dada y observaciones.
				\paragraph\indent\textbf{Lista de precios:}
				\paragraph\indent	
				Cada producto tiene asociado uno o más precios con su respectiva fecha de establecimiento. El precio vigente es el último que se haya establecido. La actualización de precios se puede realizar a todos los productos pertenecientes a un determinado grupo en igual proporción (incremento o decremento) respecto del precio vigente de cada producto. La lista de precios se puede generar en formato PDF.
				\paragraph\indent\textbf{Ventas:}
				\paragraph\indent	
				Una venta se puede realizar a partir de uno o más presupuestos existentes o no. En caso de que un cliente quiera realizar una compra de uno o más productos que hayan sido presupuestados previamente, un usuario debe buscar dicho/s presupuesto/s a través de su código de identificación o bien con datos del cliente para así acceder a sus presupuestos. De los presupuestos se puede elegir cuáles líneas de presupuesto el cliente desea comprar, pudiendo modificar la cantidad del mismo y asignar el precio a aquellos que no tuviesen, o bien el precio del producto haya cambiado. 
				\paragraph\indent
				Las líneas de presupuesto que hayan sido seleccionadas, pasan al estado de utilizados, mientras que las descartadas pasan al estado de no utilizadas. A su vez, el presupuesto utilizado pasa al estado vendido.
				\paragraph\indent
				El cliente debe realizar un pago total o parcial y se le solicita, en caso que no lo haya otorgado, su número de teléfono, una dirección e información de entrega (nombre y apellido de la persona que recibirá el o los productos en la dirección de entrega), siendo de carácter obligatorio el número de teléfono. Una venta tiene un cliente, código de identificación, una o más facturas, uno o más recibos, fecha de venta, monto total, una o más líneas de producto, una orden de producción, uno o más remitos, ubicación donde se genera la venta, un usuario asociado y un estado. Dichos datos serán obligatorios.
				\paragraph\indent
				Los posibles estados de una venta son los siguientes:
				\begin{itemize}
					\item \textbf{En creación:} Cuando no tiene líneas de producto o todas las líneas de producto se encuentran en creación.
					\item \textbf{En revisión:} Cuando se necesita que un administrador revise una venta ya que una línea de producto perteneciente a la venta no tiene el precio actual.
					\item \textbf{Pendiente:} Cuando al menos una línea de producto se encuentra en estado distinto de en creación, cancelada o entregada.
					\item \textbf{Cancelada:} Cuando todas las líneas de producto se encuentran en estado cancelada. El/Los presupuesto/s y sus respectivas líneas vuelven al estado en que se encontraban antes de realizar la venta. Además se dará de baja todas las facturas y recibos asociados, y se realizará una nota de crédito de ser necesario.
					\item \textbf{Entregada:} Cuando todas las líneas de producto no canceladas estén en estado entregadas.
				\end{itemize}
				\paragraph\indent
				Cuando todas las líneas de producto no canceladas de una venta se encuentran en estado verificada se envía un correo electrónico al cliente (en caso de haberlo proporcionado) notificándole su deuda en caso de existir e informándole que los productos que compró ya están disponibles.
				\paragraph\indent\textbf{Líneas de producto:}
				\paragraph\indent	
				Una venta tiene una o más líneas de producto. Una línea de producto está compuesta por: un producto, la cantidad solicitada de dicho producto, cero o más observaciones, el precio por la cantidad solicitada y estado. Siendo de carácter obligatorio producto, cantidad y estado.
				\paragraph\indent
				Los posibles estados de una línea de producto son:
				\begin{itemize}
					\item \textbf{En creación:} Cuando una línea de producto se está creando.
					\item \textbf{Pendiente:} Cuando la línea de producto fue creada y se debe determinar si es necesario generar una orden de producción o se utilizará productos en stock.
					\item \textbf{Cancelada:} Cuando un cliente solicita una cancelación de la línea de producto.
					\item \textbf{Pendiente de producción:} Cuando un administrador seleccionó la línea para ser producida pero aún no se está realizando ninguna de sus tareas de producción.
					\item \textbf{En producción:} Cuando se están realizando al menos una de sus tareas de producción asignadas.
					\item \textbf{Reservada:} Cuando los productos se reservaron para el cliente.
					\item \textbf{Pendiente de entrega:} Cuando la línea de producto ha sido reservada y el crédito es superior al total de dicha línea de producto. Se debe tener en cuenta que:
					\begin{center}
						CRÉDITO = MONTO TOTAL DE VENTA - (TOTAL EN PRODUCTOS ENTREGADO + DEUDA)
	
						DEUDA = MONTO TOTAL - SUMA DE RECIBOS
					\end{center}
					\item \textbf{Entregada:} Cuando una línea de producto fue entregada.
					\item \textbf{Verificada:} Cuando la línea de producto se encontraba en producción y un administrador ha verificado que su producción ha finalizado.
				\end{itemize}
				\paragraph\indent\textbf{Tareas de producción:}
				\paragraph\indent	
				Al momento de producir una línea de producto se deben realizar una serie de tareas. Las tareas pueden tener encadenada una tarea a realizarse posterior a que se ha verificado su finalización o no. Las tareas son asignadas a un fabricante y deben ser verificadas por un administrador. Los posibles estados de las tareas son:
				\begin{itemize}
					\item \textbf{Pendiente:} Cuando se crea y asigna una tarea a un fabricante.
					\item \textbf{En proceso:} Cuando el fabricante indica que ha comenzado a realizar la tarea.
					\item \textbf{Cancelada:} Cuando un administrador desea cancelar la ejecución de la tarea.
					\item \textbf{Pausada:} Cuando un administrador desea pausar la ejecución de una tarea.
					\item \textbf{Finalizada:} Cuando el fabricante indica que ha finalizado su tarea.
					\item \textbf{Verificada:} Cuando un administrador verifica que la tarea se ha finalizado de forma correcta.
				\end{itemize}
				\paragraph\indent\textbf{Facturas:}
				\paragraph\indent	
				Una venta tiene una o más facturas asociadas, de las facturas se desea saber: número de factura, venta asociada, una fecha en la que se realiza, uno o más recibos, cero o más notas de crédito y estado (pudiendo ser: alta o baja). Si una venta se encuentra en estado pendiente o entregada, debe tener una factura en estado activo. Las facturas se pueden gestionar (crear, modificar, borrar, dar de alta y dar de baja) por aquellos usuarios que cuenten con los permisos necesarios.
				\begin{center}
					MONTO TOTAL DE VENTA = SUMA TOTAL DE LOS RECIBOS ASOCIADOS A LA FACTURA ACTIVA DE DICHA VENTA - MONTO TOTAL DE LA NOTA DE CRÉDITO ASOCIADA A LA FACTURA ACTIVA DE DICHA VENTA (EN CASO DE EXISTIR)
				\end{center}
				\paragraph\indent
				Al momento de cancelar una venta se puede cancelar la factura (pasándola al estado baja) o bien asociarle una o más notas de crédito.
				\paragraph\indent\textbf{Notas de crédito:}
				\paragraph\indent	
				Una nota de crédito tiene un número de identificación, la fecha en la que se realiza, el monto de la misma, una factura asociada y un estado (pudiendo ser: alta o baja). Las notas de crédito se pueden gestionar (crear, borrar, dar de alta y dar de baja) por aquellos usuarios que cuenten con los permisos necesarios.
				\paragraph\indent\textbf{Recibos:}
				\paragraph\indent	
				Cada factura tiene uno o más recibos asociados. De los recibos interesa saber: código de identificación, fecha en que se realizó, cantidad de dinero recibido, medio de pago utilizado, factura asociada y descripción. Los recibos se pueden gestionar (crear, modificar y borrar) por aquellos usuarios que cuenten con los permisos necesarios para hacerlo.
				\paragraph\indent\textbf{Órdenes de Producción:}
				\paragraph\indent	
				Una orden de producción está compuesta por una o más líneas de producto, fecha de creación, fecha en la que se finaliza, usuario que la crea, usuario que marca como finalizada y usuario que verifica que esté finalizada y estado.
				\paragraph\indent
				Los estados de una orden de producción pueden ser:
				\begin{itemize}
					\item \textbf{En creación:} Cuando se están agregando líneas de producto a la orden de producción.
					\item \textbf{Pendiente:} Cuando las líneas de producto no canceladas, no verificadas y no entregadas de la orden de producción se encuentran en estado pendiente de producción.
					\item \textbf{En producción:} Cuando al menos una línea de producto de la orden de producción se encuentran en estado de en producción.
					\item \textbf{Cancelada:} Cuando todas las líneas de producto de la orden de producción se encuentran en estado cancelada.
					\item \textbf{Verificada:} Cuando todas las líneas de productos no canceladas se encuentran verificadas o entregadas.
				\end{itemize}
				\paragraph\indent
				Una orden de producción puede ser creada a partir de una venta existente o no. En el caso de que sea creada a partir de una venta, no se pueden agregar nuevas líneas de producto para producir.
				\paragraph\indent
				Cuando una orden de producción pase al estado de verificada se debe asignar una ubicación, de manera que todos las líneas de producto producidas se agregan al stock correspondiente de dicha ubicación.
				\paragraph\indent\textbf{Remitos:}
				\paragraph\indent	
				Un remito tiene una fecha y dirección de entrega, una o más líneas de producto y una venta asociada, por cada línea de producto, se puede seleccionar la ubicación de la cual se extraerá. Al generarse el remito se debe cambiar el estado de la línea de producto de la venta a entregado.  Decrementando, además, el stock correspondiente.
				\paragraph\indent\textbf{Órdenes de reposición:}
				\paragraph\indent	
				Los productos existentes se pueden mover de una ubicación a otra, para ello se debe generar una orden de reposición en la cual se indica la cantidad de productos a mover, ubicación origen y ubicación destino. Decrementando el stock en la ubicación origen e incrementando el stock en la ubicación destino.
		
\section{Suposiciones y dependencias}
\paragraph\indent 	
\textbf{Suposiciones:} Se asume que los requisitos en este documento son estables una vez que sean aprobados por la Dirección de Zimmerman Muebles SRL. Cualquier petición de cambios en la especificación debe ser aprobada por todas las partes intervinientes y debe ser gestionada por el equipo de desarrollo. 
\paragraph\indent
\textbf{Dependencias:} No posee dependencias.

\section{Requisitos de usuario y tecnológicos}
\textbf{Requisitos de usuario:} Los usuarios serán todas aquellas personas mayores de 18 años que puedan acceder a la aplicación Web, sean empleados de la empresa Zimmerman Muebles SRL y tengan una dirección de correo electrónico. Las interfaces deben ser intuitivas, fáciles de usar y amigables, de manera que con unas breves instrucciones sean capaces de usarla.

\textbf{Requisitos tecnológicos:} En vista de que la aplicación debe correr en diferentes navegadores de diferentes dispositivos, y teniendo en cuenta un futuro crecimiento de ZM Gestión, se optará por un entorno económico y fácil de instalar. La aplicación se ejecutará sobre un esquema Cliente/Servidor Internet, con los procesos ejecutándose parte en el servidor web y de bases de datos, y la interfaz de usuario y procesos de interfaz ejecutándose en los clientes y éstos solicitando requerimientos al servidor vía el protocolo HTTP. El navegador del cliente debe ser HTML 2.0 compatible y los servidores serán dimensionados en base a los servicios web, de bases de datos y conexiones simultáneas; la aplicación debe estar disponible en un régimen de 24x7 y el número esperado de usuarios será de 100, con un factor de simultaneidad de 40 porciento. 

\section{Requisitos de interfaces externas}
\textbf{Interfaces de usuario:} La interfaz de usuario debe ser web responsive.

\textbf{Interfaces hardware:} Pantalla 320 x 80 mínimo, teclado alfanumérico, y dispositivo señalizador o bien pantalla táctil. 

\textbf{Interfaces software:} La aplicación deberá interactuar con un sistema de envíos automático de correos electrónicos tipo MailGun.

\section{Requisitos de rendimiento}
El tiempo de respuesta de la aplicación a cada función solicitada por el usuario no debe ser superior a los 10 segundos en una velocidad efectiva de conexión de 512 Kbps. El tiempo de respuesta a los listados dependerá de la cantidad de datos solicitados.

\section{Requisitos de desarrollo}
El ciclo de vida será el de Prototipado Evolutivo, debiendo orientarse hacia el desarrollo de un sistema flexible que permita incorporar de manera sencilla cambios y nuevas funcionalidades. 

\section{Restricciones de diseño}
\paragraph\textbf{Ajuste a estándares:} No se han definido.

\paragraph\textbf{Seguridad:} La seguridad de la comunicación será establecida por https con ssh de 128 bits y la de los datos por el Sistema Gestor de Base de Datos Relacional que se emplee. 

\paragraph\textbf{Política de Respaldo:} La seguridad de la comunicación será establecida por https con ssh de 128 bits y la de los datos por el Sistema Gestor de Base de Datos Relacional que se emplee. 
\begin{itemize}
	\item 1 Backup completo por semana.
	\item 1 Backup transaccional por día. 
\end{itemize}

\paragraph\textbf{Base de Datos:} El Sistema Gestor de Base de Datos debe ser relacional.

\paragraph\textbf{Política de Borrado:} No se ha definido.

