\section{Introducción}
\paragraph\indent
El ciclo de vida es la transformación que el producto software sufre a lo largo de su vida, desde que nace hasta que muere. El resultado de las transformaciones que sufre el producto software a lo largo de su ciclo de vida representa en esencia el producto mismo y se denomina estado.
\paragraph\indent
Un ciclo de vida determina el orden de las fases de un proceso software y establece los criterios de transición de una fase a la otra. El proceso software es una colección de actividades que comienzan con la identificación de una necesidad y concluye con el retiro del software que satisface dicha necesidad. 
\paragraph\indent
Al comienzo de un proyecto software se debe elegir el ciclo de vida que seguirá el producto a construir. El modelo de ciclo de vida elegido llevará a encadenar las tareas y actividades del proceso software de una determinada manera. Se debe tener en cuenta que algunas tareas serán realizadas una vez y otras deberán realizarse más de una vez. El ciclo de vida apropiado se elige en base a la cultura de la corporación, el dominio del problema, la comprensión de los requisitos y la volatilidad de los mismos.  
\paragraph\indent
Un proyecto sin estructura es un proyecto inmanejable, no puede ser planificado ni estimado ni mucho menos alcanzar un compromiso de costos y tiempo. La idea de buscar ciclos de vida que describan las actividades a realizar para transformar el producto surgen de tener un esquema que sirvan como base para: 
\begin{itemize}
	\item Planificar
	\item Organizar
	\item Asignar personal
	\item Coordinar
	\item Presupuestar
	\item Dirigir
\end{itemize}