\section{Selección de un modelo de ciclo de vida}
\paragraph\indent
El ciclo de vida para el desarrollo del sistema de gestión de la mueblería es el de prototipado evolutivo. Para la selección del mismo se tuvo en cuenta tanto la necesidad de brindar a los usuarios las características del nuevo sistema a medida que estas sean implementadas, como así también las partes volátiles del proyecto.
\paragraph\indent
Un prototipo es un sistema a escala que tiene las características del sistema disminuidas. Se usa cuando el cliente no tiene una idea muy detallada de lo que necesita o el ingeniero en software no está muy seguro de la viabilidad de la solución.
\paragraph\indent
Las evaluaciones del usuario retroalimentan el proceso para refinar los diseños y especificaciones del sistema emergente. Los prototipos pueden además utilizarse para verificar la viabilidad del diseño del sistema y como una herramienta iterativa del desarrollo de software donde el prototipo evoluciona hasta llegar al sistema final.
\paragraph\indent
Los prototipos evolutivos son fácilmente modificables y ampliables. Una vez definidos estos requisitos, el prototipo evolucionará hasta el sistema final. Un prototipo evolutivo tiene como característica que sigue el ciclo de vida estándar, pero con el tiempo de desarrollo bastante reducido y, además, la aplicación de estándares no es muy rigurosa.
\paragraph\indent
El modelo de desarrollo basado en prototipos es una versión modificada del modelo en cascada o clásico con el fin de contrarrestar las limitaciones que este posee. Las fases del ciclo de vida clásico quedan modificadas de la siguiente manera debido a la introducción del uso de prototipos:
\begin{itemize}
	\item Análisis preliminar y especificación de requisitos de usuarios: En esta fase se hace un primer análisis de las necesidades del usuario, especificaciones generales del sistema y estudios de viabilidad. Estas especificaciones preliminares forman la base sobre la que se apoya el diseño y la implementación del prototipo. 
	\item Diseño, desarrollo e implementación del prototipo: Lo importante al desarrollar el prototipo es que su implantación sea rápida y su coste de desarrollo sea bajo. Existe una serie de factores que deben ser tenidos en cuenta para conseguir dichos objetivos:
	\begin{itemize}
		\item Énfasis en la interfaz del usuario, la cual debe permitir completar los requerimientos después del desarrollo.
		\item Desarrollar el prototipo con un pequeño equipo de desarrollo que minimice los problemas de comunicación y sean los mismos que desarrollaron las especificaciones.
		\item Utilizar un lenguaje adecuado para el desarrollo del prototipo, que permita una rápida detección de errores y facilidades de manipulación de datos.
		\item Buscar las herramientas adecuadas para un desarrollo rápido.
  \end{itemize}
	\item Prueba del prototipo: Es importante desarrollar adecuadamente esta actividad y extraer el máximo de conclusiones de la experiencia de los usuarios en el uso del prototipo. Como elementos claves se debe señalar:
	\begin{itemize}
		\item La asignación de un tiempo suficiente a la actividad de planificación del desarrollo, como para que los usuarios puedan probar el prototipo y comunicar sus experiencias.
		\item Los departamentos de usuarios deben comprometerse a probar adecuadamente dicho prototipo.
		\item Planificar la formación y entrenamiento de los usuarios en el uso de prototipos.
		\item Desarrollar metodologías para recoger las impresiones de los usuarios.
  \end{itemize}
	\item Refinamiento iterativo del prototipo: Con la información proporcionada por el usuario, debe modificarse el prototipo en forma rápida para que pueda ser probado nuevamente por los usuarios. Esta experiencia y refinamiento se continúa hasta alcanzar el estado donde los beneficios de mejorar aún más el prototipo sean menores que el tiempo y costo requerido para tales modificaciones.
	\item Refinamiento de las especificaciones de requisitos: Toda la información aportada por los usuarios se analiza, y a partir de la misma se revisan las especificaciones de requisitos. Sobre ellas se procede al diseño e implementación del sistema de producción.
	\item Diseño e implantación del sistema de producción: Para ello se sigue el modelo clásico de V, al que se habrá aportado una gran intuición acerca de cómo se debería desarrollar el sistema real. 
\end{itemize}
\paragraph\indent
Existen distintos modelos de prototipos, se tienen los prototipos desechables, las maquetas y los prototipos evolutivos, este último es el que se utilizará. Es un modelo de trabajo del sistema propuesto, que aporta a los usuarios una representación física de las partes claves del sistema antes de la implantación. Es fácilmente modificable y ampliable, y una vez definidos todos los requisitos, el prototipo evolucionará hacia el sistema final. Es allí donde se implantan primero aquellos requisitos y necesidades que son claramente entendidos, y para aquellos que son críticos se realiza un diseño y análisis en detalle.