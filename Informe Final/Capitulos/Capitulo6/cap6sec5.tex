\section{Herramientas de documentación}
	
		\begin{itemize}
			\item \textbf{LaTeX:} es un sistema de composición de textos, orientado a la creación de documentos escritos que presenten una alta calidad tipográfica. Por sus características y posibilidades, es usado de forma especialmente intensa en la generación de artículos y libros científicos que incluyen, entre otros elementos, expresiones matemáticas.
			LaTeX está formado por un gran conjunto de macros de TeX, escrito por Leslie Lamport en 1984, con la intención de facilitar el uso del lenguaje de composición tipográfica, TeX, creado por Donald Knuth.
				\begin{itemize}
				\item \textbf{MiKTeX:} es una distribución TeX/LaTeX para Microsoft Windows que fue desarrollada por Christian Schenk.
				Las características más apreciables de MiKTeX son su habilidad de actualizarse por sí mismo descargando nuevas versiones de componentes y paquetes instalados previamente, y su fácil proceso de instalación.
				\item \textbf{TeXnicCenter:} es un editor software libre de LaTeX para Windows, el cual integra en sí mismo las herramientas necesarias para la composición de texto científico, desde una ventana de compilación integrada, una completa ayuda y manual de LaTeX para los usuarios primerizos, así como un entorno personalizable para los usuarios avanzados.
				\end{itemize}
			
			\item \textbf{PlantUML:} es una herramienta de código abierto que permite a los usuarios crear diagramas UML a partir de un lenguaje de texto plano. Utiliza el software Graphviz para diseñar sus diagramas.
		\end{itemize}
		
	\section{Código}
	El código fuente se encuentra en el CD adjunto.